\hypertarget{scientificComputing}{}\section{Scientific computing}\label{scientificComputing}
\hyperlink{scientificComputing_generalSoftwares}{General Softwares and Libraries for Scientific Computing} \par
 \hyperlink{scientificComputing_scc}{Scientific Computing in C++} \par
 \hyperlink{scientificComputing_scp}{Scientific Computing in Python} \par
 \hyperlink{scientificComputing_sdns}{Specific Developments for Non Smooth Systems} \par
\hypertarget{scientificComputing_generalSoftwares}{}\subsection{General Softwares and Libraries for Scientific Computing}\label{scientificComputing_generalSoftwares}
\begin{enumerate}
\item \href{http://scilinux.sourceforge.net/}{\tt {\bf Sci\-Linux}: Environment for Scientific Computing on GNU/Linux}\item \href{http://www.netlib.org/lapack/}{\tt LAPACK},\href{http://www.netlib.org/scalapack/}{\tt http://www.netlib.org/scalapack}, \href{http://math-atlas.sourceforge.net/}{\tt http://www.netlib.org/linalg}\item \href{http://math-atlas.sourceforge.net/}{\tt Automatically Tuned Linear Algebra Software ({\bf ATLAS})}\item \href{http://sources.redhat.com/gsl/}{\tt {\bf GSL} - The GNU Scientific Library, a free numerical library for C {\bf ...}}\item \href{http://www.nr.com/}{\tt {\bf Numerical} {\bf Recipes} Home Page}\item \href{http://www.ginac.de/CLN/}{\tt {\bf CLN} Class Library for Numbers}\item \href{http://linal.sourceforge.net/LinAl/Doc/linal.html}{\tt {\bf Lin\-Al} home page}\item \href{http://sal.kachinatech.com/}{\tt {\bf Scientific Applications on Linux (SAL)} } is a collection of information and links to software of interest to scientists and engineers.\item \href{http://www.numis.northwestern.edu/ftp/pub/list-packages.html}{\tt GAMS}\item \href{http://www.mathtools.net/}{\tt {\bf Mathtools}.net: The technical computing portal}\item \href{http://www.netlib.org/utk/people/JackDongarra/la-sw.html}{\tt Freely available software for linear Algebra on the Web a survey}\end{enumerate}
\hypertarget{scientificComputing_scc}{}\subsection{Scientific Computing in C++}\label{scientificComputing_scc}
\begin{enumerate}
\item \href{http://oonumerics.org/oon}{\tt The {\bf Object Oriented Numerics Page}}\item \href{http://gams.nist.gov/lapack++/}{\tt {\bf LAPACK++}: Linear Algebra Package in C++}\item \href{http://www-hpc.jpl.nasa.gov/PEP/nortonc/thesis.html}{\tt Object Oriented Programming Paradigms in Scientific Computing}\item \href{http://www.oonumerics.org/blitz/}{\tt Blitz++ Home Page}\item \href{http://www.gmm.insa-tlse.fr/getfem/gmm.html}{\tt {\bf GMM++} presentation page}\item \href{http://gams.nist.gov/tnt/}{\tt {\bf TNT} Home Page}, an interface for scientific computing in C++. It provides a distinction between interfaces and implementations of {\bf TNT} components.\item \href{http://www.osl.iu.edu/research/mtl/}{\tt The {\bf Matrix Template Library}}\end{enumerate}
\hypertarget{scientificComputing_scp}{}\subsection{Scientific Computing in Python}\label{scientificComputing_scp}
\begin{enumerate}
\item \href{http://www.scipy.org}{\tt Sci\-Py.org, }an open source library of scientific tools for Python. Sci\-Py supplements the Numeric module, gathering a variety of high level science and engineering modules together as a single package. Within Sci\-Py are modules for graphics and plotting, optimization, integration, special functions, signal and image processing, genetic algorithms, ODE solvers, and others. There is also an experimental \char`\"{}compiler\char`\"{} that takes a Numeric array expression in Python and compiles it to C++ code on the fly.\item \href{http://www.swig.org/}{\tt SWIG : Simplified Wrapper and Interface Generator} {\bf SWIG} is a software development tool that connects programs written in C and C++ with a variety of high-level programming languages. {\bf ...}\item \href{http://www.pfdubois.com/numpy/}{\tt Numerical {\bf Python}} Numerical Python adds a fast, compact, multidimensional array language facility to Python.\item \href{http://starship.python.net/crew/hinsen/scientific.html}{\tt Scientific Python} Konrad Hinsen's \href{http://starship.python.net/crew/hinsen/scientific.html}{\tt Scientific Python} is a module library for scientific computing. In this collection you will find modules that cover basic geometry (vectors, tensors, transformations, vector and tensor fields), quaternions, automatic derivatives, (linear) interpolation, polynomials, elementary statistics, nonlinear least-squares fits, unit calculations and conversions, Fortran-compatible text formatting, 3D visualization via VRML, two Tk widgets for simple line plots and 3D wireframe models. Scientific Python also contains Python interfaces to the net\-CDF library (implementing a portable binary format for large arrays) and the Message Passing Interface, the most widely used communications library for parallel computers. Konrad Hinsen's course, \href{http://dirac.cnrs-orleans.fr/%7Ehinsen/courses.html}{\tt Python for Scientists} shows how to use \href{http://starship.python.net/crew/hinsen/scientific.html}{\tt Scientific Python}.\item Pyfort, a Python / Fortran connection tool. See \href{http://pyfortran.sourceforge.net}{\tt Pyfort Home Page}\item FPIG, a Python / Fortran connection tool. See \href{http://cens.ioc.ee/projects/f2py2e}{\tt FPIG Home Page}.\end{enumerate}
\hypertarget{scientificComputing_sdns}{}\subsection{Specific Developments for Non Smooth Systems}\label{scientificComputing_sdns}
None of the links of this page correponds to a member of the project.\hypertarget{scientificComputing_sdns_1}{}\subsubsection{General theory for non smooth systems}\label{scientificComputing_sdns_1}
\hypertarget{scientificComputing_sdns_2}{}\subsubsection{Complementarity problems and Variational inequalities.}\label{scientificComputing_sdns_2}
\begin{enumerate}
\item \href{http://www.cs.wisc.edu/cpnet/}{\tt CPNET: Complementarity Problem Net}\item \href{http://plato.la.asu.edu/topics/problems/mcp.html}{\tt Complementarity Problems}\item \href{http://www.cs.wisc.edu/%7Eferris/}{\tt Professor Michael C. Ferris}\item \href{http://www.stanford.edu/dept/MSandE/faculty/rwc/}{\tt Richard W. Cottle}\item \href{http://www.mts.jhu.edu/%7Epang/}{\tt Jong-Shi Pang's Home Page}\end{enumerate}
\hypertarget{scientificComputing_sdns_3}{}\subsubsection{Optimization \& Mathematical programming}\label{scientificComputing_sdns_3}
\begin{enumerate}
\item \href{http://www.cs.wisc.edu/%7Eswright/}{\tt Steve Wright}\item \href{http://www-neos.mcs.anl.gov/}{\tt {\bf NEOS} Server for Optimization} - The NEOS Server solvers represent the state-of-the-art in optimization software.\item \href{http://www-fp.mcs.anl.gov/otc/Guide/softwareGuide/}{\tt Optimization Software}: linear and nonlinear programming.\item \href{http://www.ampl.com/}{\tt {\bf AMPL} Modeling Language for Mathematical Programming}, modeling language and system for formulating, solving and analyzing large-scale optimization (mathematical programming) problems.\end{enumerate}
\hypertarget{scientificComputing_sdns_4}{}\subsubsection{Numerical time integration}\label{scientificComputing_sdns_4}
\begin{enumerate}
\item \href{http://www.math.uiowa.edu/%7Edstewart/}{\tt David Stewart}\end{enumerate}
\hypertarget{scientificComputing_sdns_5}{}\subsubsection{Frictional contact mechanical systems}\label{scientificComputing_sdns_5}
\hypertarget{scientificComputing_sdns_6}{}\subsubsection{Electrical systems}\label{scientificComputing_sdns_6}
