\newcommand{\fcadre}[1]
        {\begin{center}\fbox{$ \displaystyle {#1} $} \end{center}}
\newcommand{\Fcadre}[1]
        {\fbox{$ \displaystyle {#1} $} }        
\newcommand{\w}[1]{\omega_{#1}}
\newcommand{\brw}[1]{\Breve{\omega}_{#1}}
\newcommand{\brgam}[2]{\Breve{\gamma}_{x_{#1}x_{#2}}}
\newcommand{\bru}[1]{\Breve{u}_{#1}}
\newcommand{\bPI}{\boldsymbol\Pi}
%Operateur de derivee partielle classique
\newcommand{\DP}[2]{\displaystyle \frac{\partial {#1}}{\partial {#2}}}
\newcommand{\DPP}[2]{\displaystyle\frac{\partial^2 {#1}}{\partial {#2}^2}}
%Operateur de derivee partielle d'une parenthese separee
\newcommand{\DS}[2]{\displaystyle\frac{\partial}{\partial {#2}}\left(#1\right)}
\newcommand{\DSS}[2]{\displaystyle\frac{\partial^2}{\partial {#2}^2}\left(#1\right)}
%Operateur de derivee  classique
\newcommand{\DT}[2]{\displaystyle \frac{d\, {#1}}{d\,{#2}}}
\newcommand{\DTT}[2]{\displaystyle\frac{d^2\, {#1}}{d\, {#2}^2}}



\newcommand{\intinf}[1]{\int_{-\infty}^{+\infty} {#1}\,dt}
\newcommand{\INTA}[1]{\int^{A}_{-A}{#1}\,dx}
\newcommand{\INTL}[1]{\int^{L}_{0}{#1}\,dx_{1}}
\newcommand{\DBINT}[1]{\int \int_{\Omega}{#1}\,dx\,dt}
\newcommand{\INTS}[1]{\int_{S} {#1} \,ds}
\def\dx{\,dx}
\def\dt{\,dt}
\def\ds{\,ds}
\def\dnu{\,d\nu}
\def\dtheta{\,d\theta}


\newcommand{\norme}[1]{\|#1\|}
\newcommand{\normeinf}[1]{\|#1\|_{\infty}}
\newcommand{\normeE}[1]{\|#1\|_{\mathcal E}}
\newcommand{\Lnorme}[1]{\norme{#1}_{{Cal L}_2}} 
\newcommand{\scal}[2]{\left\langle{#1},{#2}\right\rangle} 





\newcommand{\etoi}[1]{\stackrel{\star}{#1}}
\newcommand{\contract}{{\,\Bar{\Bar\otimes}\,}}
%\newcommand{\contract}{{\,\overline{\overline\otimes}\,}}
\newcommand{\scontract}{\,{\Bar\otimes}\,}
\newcommand{\ouvert}[1]{\stackrel{\circ}{#1}}
% displaystyle et fraction displayed

%% Symbole de fraction
\newcommand{\Frac}[2]{{\displaystyle \frac{\displaystyle #1}{\displaystyle #2}}}
\newcommand{\Prac}[2]{\displaystyle \genfrac{(}{)}{}{}{\displaystyle #1}{\displaystyle #2}}
\newcommand{\Crac}[2]{\displaystyle \genfrac{[}{]}{}{}{\displaystyle #1}{\displaystyle #2}}
%\newcommand{\binom}[2]{\genfrac{(}{)}{0pt}{}{#1}{#2}}
%\def\bigint{{\displaystyle\int}} 
\def\Int{\displaystyle\int}
%\newcommand{\Int}{\displaystyle\int}




% Symboles mathemeatiques en gras
\newcommand{\baa}{\boldsymbol a }
\newcommand{\baA}{\boldsymbol A }
\newcommand{\bbb}{\boldsymbol b }
\newcommand{\bbB}{\boldsymbol B }
\newcommand{\bcC}{\boldsymbol C }
\newcommand{\bcc}{\boldsymbol c }
\newcommand{\bdD}{\boldsymbol D }
\newcommand{\bdd}{\boldsymbol d }
\newcommand{\bee}{\boldsymbol e }
\newcommand{\beE}{\boldsymbol E }
\newcommand{\bff}{\boldsymbol f }
\newcommand{\bfF}{\boldsymbol F }
\newcommand{\bgg}{\boldsymbol g }
\newcommand{\bgG}{\boldsymbol G }
\newcommand{\bhh}{\boldsymbol h }
\newcommand{\bhH}{\boldsymbol H }
\newcommand{\bii}{\boldsymbol i }
\newcommand{\biI}{\boldsymbol I }
\newcommand{\bll}{\boldsymbol l }
\newcommand{\blL}{\boldsymbol L }
\newcommand{\bmm}{\boldsymbol m}
\newcommand{\bmM}{\boldsymbol M}
\newcommand{\bnn}{\boldsymbol n }
\newcommand{\bnN}{\boldsymbol N}
\newcommand{\bkK}{\boldsymbol K }

\newcommand{\bpp}{\boldsymbol p}
\newcommand{\bpP}{\boldsymbol P}
\newcommand{\bqq}{\boldsymbol q}
\newcommand{\bddq}{\Ddot{\boldsymbol q}}
\newcommand{\bdq}{\Dot{\boldsymbol q}}
\newcommand{\bqQ}{\boldsymbol Q}
\newcommand{\brR}{\boldsymbol R }
\newcommand{\brr}{\boldsymbol r }
\newcommand{\bss}{\boldsymbol s }
\newcommand{\bsS}{\boldsymbol S }
\newcommand{\btt}{\boldsymbol t }
\newcommand{\btT}{\boldsymbol T }
\newcommand{\buu}{\boldsymbol u}
\newcommand{\buU}{\boldsymbol U}
\newcommand{\bvV}{\boldsymbol V}
\newcommand{\bvv}{\boldsymbol v}
\newcommand{\bhvv}{\Hat{\boldsymbol v}}
\newcommand{\bwW}{\boldsymbol W}
\newcommand{\bww}{\boldsymbol w}
\newcommand{\bxX}{\boldsymbol X }
\newcommand{\bxx}{\boldsymbol x }
\newcommand{\byY}{\boldsymbol Y }
\newcommand{\byy}{\boldsymbol y }
\newcommand{\bzZ}{\boldsymbol Z }
\newcommand{\bzz}{\boldsymbol z }

\newcommand{\bdel}{\boldsymbol \delta }
\newcommand{\brho}{\boldsymbol \rho }
\newcommand{\bgam}{\boldsymbol \gamma }
\newcommand{\bsig}{\boldsymbol \sigma }
\newcommand{\beps}{\boldsymbol \epsilon }
\newcommand{\btau}{\boldsymbol \tau }
\newcommand{\btheta}{\boldsymbol \theta }
\newcommand{\balpha}{\boldsymbol \alpha }
\newcommand{\bbeta}{\boldsymbol \beta }
\newcommand{\bomega}{\boldsymbol \omega}
\newcommand{\bOmega}{\boldsymbol \Omega}
\newcommand{\bups}{\boldsymbol \Upsilon}

\newcommand{\mC}{\mathcal C}
\newcommand{\mR}{\mathcal R}
\newcommand{\HRule}{\rule{\linewidth}{1mm}}
\newcommand{\demi}{{\displaystyle\frac{1}{2}}}
\newcommand{\ddemi}{\frac{1}{2}}
\newcommand{\tiers}{\displaystyle\frac{1}{3}}
\newcommand{\quart}{\displaystyle\frac{1}{4}}
\newcommand{\dixi}{\displaystyle\frac{1}{10}}
\newcommand{\RR}{\mathbb R}
\newcommand{\bsy}[1]{\boldsymbol{#1}}


% Fonction math�matiques

\newcommand{\divx}{\text{div}_{\boldsymbol x}}
\newcommand{\ddivx}{\text{divdiv}_{\boldsymbol x}}
\newcommand{\divX}{\text{div}_{\boldsymbol X}}
\newcommand{\gradx}{\text{grad}_{\boldsymbol x}}
\newcommand{\sign}{\text{sign}}
\newcommand{\divy}{\text{div}_{\boldsymbol y}}
\newcommand{\transposee}[1]{{\vphantom{#1}}^{\text{\tiny{\textsf T}}}{#1}}
%\newcommand{\transposee}[1]{{\vphantom{#1}}^{\mathit{t}}{#1}}
\newcommand{\invtransposee}[1]{{\vphantom{#1}}^{\text{T}}{#1}}
%\newcommand{\invtransposee}[1]{{\vphantom{#1}}^{\mathit{-t}}{#1}}
\newcommand{\argmin}{\mathop{\mathrm{argmin}}}
\newcommand{\argminn}{\mathop{\mathrm{argmin}}\nolimits}
% macro pour les symbols d'ensemble
%\nbOne
\def\nbOne{{\mathchoice{\rm 1\mskip-4mu l}{\rm 1\mskip-4mu l} {\rm 1 \mskip-4.5mu l}{\rm 1\mskip-5mu l}}}
%
%%  Les ensembles de nombres  C. Fiorio (fiorio�at�math.tu-berlin.de) 
%
\def\nbR{\ensuremath{\mathrm{I\!R}}} % IR
\def\nbN{\ensuremath{\mathrm{I\!N}}} % IN
\def\nbF{\ensuremath{\mathrm{I\!F}}} % IF
\def\nbH{\ensuremath{\mathrm{I\!H}}} % IH
\def\nbK{\ensuremath{\mathrm{I\!K}}} % IK
\def\nbL{\ensuremath{\mathrm{I\!L}}} % IL
\def\nbM{\ensuremath{\mathrm{I\!M}}} % IM
\def\nbP{\ensuremath{\mathrm{I\!P}}} % IP
%
% \nbOne : 1I : symbol one
\def\nbOne{{\mathchoice {\rm 1\mskip-4mu l} {\rm 1\mskip-4mu l}
{\rm 1\mskip-4.5mu l} {\rm 1\mskip-5mu l}}}
%
% \nbC   :  Nombres Complexes
\def\nbC{{\mathchoice {\setbox0=\hbox{$\displaystyle\rm C$}%
\hbox{\hbox to0pt{\kern0.4\wd0\vrule height0.9\ht0\hss}\box0}}
{\setbox0=\hbox{$\textstyle\rm C$}\hbox{\hbox
to0pt{\kern0.4\wd0\vrule height0.9\ht0\hss}\box0}}
{\setbox0=\hbox{$\scriptstyle\rm C$}\hbox{\hbox
to0pt{\kern0.4\wd0\vrule height0.9\ht0\hss}\box0}}
{\setbox0=\hbox{$\scriptscriptstyle\rm C$}\hbox{\hbox
to0pt{\kern0.4\wd0\vrule height0.9\ht0\hss}\box0}}}}
%
% \nbQ   : Nombres Rationnels Q
\def\nbQ{{\mathchoice {\setbox0=\hbox{$\displaystyle\rm
Q$}\hbox{\raise
0.15\ht0\hbox to0pt{\kern0.4\wd0\vrule height0.8\ht0\hss}\box0}}
{\setbox0=\hbox{$\textstyle\rm Q$}\hbox{\raise
0.15\ht0\hbox to0pt{\kern0.4\wd0\vrule height0.8\ht0\hss}\box0}}
{\setbox0=\hbox{$\scriptstyle\rm Q$}\hbox{\raise
0.15\ht0\hbox to0pt{\kern0.4\wd0\vrule height0.7\ht0\hss}\box0}}
{\setbox0=\hbox{$\scriptscriptstyle\rm Q$}\hbox{\raise
0.15\ht0\hbox to0pt{\kern0.4\wd0\vrule height0.7\ht0\hss}\box0}}}}
%
% \nbT   : T
\def\nbT{{\mathchoice {\setbox0=\hbox{$\displaystyle\rm
T$}\hbox{\hbox to0pt{\kern0.3\wd0\vrule height0.9\ht0\hss}\box0}}
{\setbox0=\hbox{$\textstyle\rm T$}\hbox{\hbox
to0pt{\kern0.3\wd0\vrule height0.9\ht0\hss}\box0}}
{\setbox0=\hbox{$\scriptstyle\rm T$}\hbox{\hbox
to0pt{\kern0.3\wd0\vrule height0.9\ht0\hss}\box0}}
{\setbox0=\hbox{$\scriptscriptstyle\rm T$}\hbox{\hbox
to0pt{\kern0.3\wd0\vrule height0.9\ht0\hss}\box0}}}}
%
% \nbS   : S
\def\nbS{{\mathchoice
{\setbox0=\hbox{$\displaystyle     \rm S$}\hbox{\raise0.5\ht0%
\hbox to0pt{\kern0.35\wd0\vrule height0.45\ht0\hss}\hbox
to0pt{\kern0.55\wd0\vrule height0.5\ht0\hss}\box0}}
{\setbox0=\hbox{$\textstyle        \rm S$}\hbox{\raise0.5\ht0%
\hbox to0pt{\kern0.35\wd0\vrule height0.45\ht0\hss}\hbox
to0pt{\kern0.55\wd0\vrule height0.5\ht0\hss}\box0}}
{\setbox0=\hbox{$\scriptstyle      \rm S$}\hbox{\raise0.5\ht0%
\hboxto0pt{\kern0.35\wd0\vrule height0.45\ht0\hss}\raise0.05\ht0%
\hbox to0pt{\kern0.5\wd0\vrule height0.45\ht0\hss}\box0}}
{\setbox0=\hbox{$\scriptscriptstyle\rm S$}\hbox{\raise0.5\ht0%
\hboxto0pt{\kern0.4\wd0\vrule height0.45\ht0\hss}\raise0.05\ht0%
\hbox to0pt{\kern0.55\wd0\vrule height0.45\ht0\hss}\box0}}}}
%
% \nbZ   : Entiers Relatifs Z
\def\nbZ{{\mathchoice {\hbox{$\sf\textstyle Z\kern-0.4em Z$}}
{\hbox{$\sf\textstyle Z\kern-0.4em Z$}}
{\hbox{$\sf\scriptstyle Z\kern-0.3em Z$}}
{\hbox{$\sf\scriptscriptstyle Z\kern-0.2em Z$}}}}
%%%% fin macro %%%%



\newcommand{\putidx}[1]{\index{#1}\textit{#1}}
% macro pour r�f�rencer les �quations

\newcommand{\refeq}[1]{(\ref{#1})}
\newcommand{\reffig}[1]{({\it cf} figure : \ref{#1})}
\newcommand{\refann}[1]{({\it cf} Annexe : \ref{#1})}


%\definecolor{darkgray}{gray}{.25}
%\definecolor{gray}{gray}{.5}
%\definecolor{lightgray}{gray}{.75}
%\definecolor{gradbegin}{rgb}{0,1,1}
%\definecolor{gradend}{rgb}{0,.1,.95}
%\newcommand{\newtexte}[1]{\textcolor{darkgray} {#1}}
\newcommand{\newtexte}[1]{{#1}}% macro pour les varibales favorites
% normal tangent
\def\n{{\hbox{\tiny{N}}}}
\def\t{{\hbox{\tiny{T}}}}
\def\nt{\hbox{\tiny{NT}}}
\def\nsf{\hbox{\tiny{\textsf N}}}
\def\tsf{\hbox{\tiny{\textsf T}}}
\def\sigman{\sigma_{\n}}
\def\sigmat{\sigma_{\t}}
\def\sigmant{\sigma_{\nt}}
\def\epsn{\epsilon_{\n}}
\def\epst{\epsilon_{\t}}
\def\epsnt{\epsilon_{\nt}}
\def\eps{\epsilon}
\def\veps{\varepsilon}
\def\sig{\sigma}
\def\Rn{R_{\n}}
\def\Rt{R_{\t}}
\def\cn{c_{\n}}
\def\Cn{C_{\n}}
\def\ct{c_{\t}}
\def\Ct{C_{\t}}
\def\un{u_{\n}}
\def\ut{\buu_{\t}}
\def\uut{u_{\t}}
\def\unc{u_{\n}^c}
\def\utc{\buu_{\t}^c}
\def\vn{v_{\n}}
\def\vt{v_{\t}}
\def\rr{\hbox{\tiny{\textsf R}}}
\def\irr{\hbox{\tiny{\textsf{IR}}}}
\def\rn{r_{\n}}
\def\rt{\brr_{\t}}
\def\rnc{r_{\n}^c}
\def\rtc{\brr_{\t}^c}
\def\trn{\Tilde{r}_{\n}}
\def\trt{\Tilde{\brr}_{\t}}
\def\tr{\Tilde{\brr}}
\def\tv{\Tilde{\bvv}}
\def\vn{v_{\n}}
\def\vt{\bvv_{\t}}
\def\adh{\mathsf{adh}}
\def\adj{\hbox{\tiny{\textsf{adj}}}}
\def\adjc{\hbox{\tiny{\textsf{adjC}}}}
\def\adja{\hbox{\tiny{\textsf{adjA}}}}
\def\cc{\hbox{\tiny{\textsf C}}}
\def\ca{\hbox{\tiny{\textsf A}}}
%%    Unit�e
\def\mm{\,\mathsf{mm}}
\def\cm{\,\mathsf{cm}}
\def\m{\,\mathsf{m}}
\def\ms{\,\mathsf{m.s^{-1}}}
\def\mms{\,\mathsf{mm.s^{-1}}}
\def\Mpa{\,\mathsf{MPa}}
\def\Gpa{\,\mathsf{GPa}}
\def\Kg{\,\mathsf{Kg}}
\def\Hz{\,\mathsf{Hz}}
\def\kHz{\,\mathsf{kHz}}
\def\N{\,\mathsf{N}}
\def\kN{\,\mathsf{kN}}
\def\Nmmm{\,\mathsf{N.m^{-3}}}
\def\ds{d_{\hbox{\tiny{S}}}}
% domaines et frontieres
\def\om{\Omega}
\def\oma{\Omega^{\alpha}}
\def\omu{\Omega^1\cup \Omega^2}
\def\gc{\Gamma_c}
\def\omt{\omu \cup \gc}
% derivee partielle et gradient et divergence
\def\p{\partial}
\def\grad{\nabla}
\def\div{\mathop{\rm div}\nolimits}
%

%\DeclareTextSymbol{\deg}{T1}{6}
%\def\degre{\mathdegree}
%\newcommand{\degre}{\mathdegree}

\def\etc{\textit{etc}\ldots}
\newcommand{\mdegre}{\hbox{\text{\degre}}}

%\def\nscd{\textsf{\bfseries NSCD}}
%\def\nscd{\textsf{NSCD}}
\newcommand{\nscd}{\textsf{NSCD}}
%\Pisymbol{psy}{212} ou encore \Pisymbol{psy}{228}

\DeclareMathOperator{\arcsh}{arcsh}
\DeclareMathOperator{\argth}{argth}
\DeclareMathOperator{\rot}{rot}


%%The Principal Value Integral symbol
\def\Xint#1{\mathchoice
   {\XXint\displaystyle\textstyle{#1}}%
   {\XXint\textstyle\scriptstyle{#1}}%
   {\XXint\scriptstyle\scriptscriptstyle{#1}}%
   {\XXint\scriptscriptstyle\scriptscriptstyle{#1}}%
   \!\int}
\def\XXint#1#2#3{{\setbox0=\hbox{$#1{#2#3}{\int}$}
     \vcenter{\hbox{$#2#3$}}\kern-.5\wd0}}
\def\ddashint{\Xint=}
\def\dashint{\Xint-}


%----------------------------------------------------------------------
%             Des chiffres avec des ronds autour
%----------------------------------------------------------------------
\def\nombrecercle#1{\def\taille{0.3}
                \put(0,0){#1}
                \put(0.08,0.08){\circle{\taille}}}

%%


%%% Local Variables: 
%%% mode: latex
%%% TeX-master: "these"
%%% x-symbol-coding: iso-8859-2
%%% End: 
