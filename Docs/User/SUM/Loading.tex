
This chapter allows to know how to launch the platform with the needed data.
It is important to understand that all the users can do correspond to the API's methods.

\section{Loading from an XML file}
It is the simpliest way to use the platform. You only need to write a correct XML file (see \ref(XML)) and to call the XML file loading function. To do that, follow the next
steps (the example given correspond to C++ language):
\begin{itemize}
	\item Declaration of a Model : \textbf{Model sample\_model;}
	\item Loading an XML file : \textbf{sample\_model.createModel("xml\_input\_file");}
\end{itemize}


\section{Loading in a C++ program}
We understand here that no file exists to load input data. The user must do these operations manually. The following example (in C++) show the way to create the platform :
\begin{itemize}
	\item Declaration of a Model : \textbf{Model sample\_model;}
	\item Creation of the Model : \textbf{sample\_model.createModel(NULL}/*the XML file*/\textbf{, 0}/*current time t*/\textbf{, 0}/*initial time t0*/\textbf{, 10}/*final time T*/\textbf{);}
	\item Creation of the NSDS : \textbf{NSDS* nsds = m.createNSDS( false} \textbf{/*determines if the NSDS is BVP or not*/);}
	\item Add of a new dynamical system to the NSDS : \textbf{nsds->addNonLinearSystemDS( 1}/*the number of the dynamical system*/\textbf{, 2}/*the number of dimension of
	the system*/\textbf{, x0}/*the x0 vector*/\textbf{, "BasicPlugin:vectorField"}/*the vector field plugin*/\textbf{ );}
	\item \dots
\end{itemize}


\section{Loading using the two previous methods}
It consists in using an XML file to load some of the data and to give more data in a program. For example, tha XML file can bring the data of the Model and all the model
formalization informations (the data of the NSDS, the different dynamical systems, the interactions, \dots), whereas strategy informations are given at the continuation of the
program (the TimeDiscretisation, the integrators, the OneStepNSProblem).
