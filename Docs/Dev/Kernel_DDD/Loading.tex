\section{Model Loading Strategy}
\label{Sec:LoadingStrategy}


\subsection{Model loading from an  XML data file}

The reading of the file generate a DOM tree in memory with all the data. After that,  the creation of the SiconosModelXML tree is  based on this DOM tree.  The SiconosModelXML belongs to the SiconosModel object, which is the main object of the platform. From the SiconosModel,  the creation process is launched to build  all of the model. From this point, the Model creates the NSDS, which one creates the dynamical systems, ... The building is "top down" and gradually, with the Model at the top of the platform. 





\subsection{Creating and loading model through the API}

It is also  possible to create the objects of the platform without a XML file by using the API of the platform. The methods given to the users allow the creation of each object of the platform thanks to the constructors of each objects.


\subsection{Mixed strategies}


\section{Definition of three major types of construtors for the Siconosmodel objects.}

%% \subsection{The "create" methods}
%% The "create" functions have for goal to initialize the relating object. They fill its fields with XML data and link the objects belonging to it to their corresponding XML object, or, if the platform is manually built, only fill its fields with the data given in paramaters.



\subsection{Default Constructor {\tt Object::Object() } }

The default constructor performs the following operations :
\begin{verbatim}
Object::Object()
   {
   this->type = type ;       // Type definition 
   this->objectXML = NULL ;  // ObjectXML pointer

   // Default Constructor (if any) and/or  initialization to default Values
   //  of the other attributes

   // Simple Type Attribute
   attribute1 = DefaultvalueAttribute1();
   ....
   attributeN = DefaultvalueAttributeN();

   //Object attributes : call  the default constructor of each attributes, 
   // with new if needed
   attributeObject1 = new AttributeObject1type();
   ...
   attributeObjectN = new AttributeObjectNtype();
   }
\end{verbatim} 

\begin{ndr}
  \begin{enumerate}
  \item Remove init() method !!!! This is the role of the Default constructor.
  \item Do we need to use the new operator if the sizes of the object are not defined ? What is the interest to call the Default constructor with a new ?
  \end{enumerate}

\end{ndr}

\subsection{Constructor from the mininal data {\small \tt Object::Object(AttributeType1 data1,...,AttributeTypeN dataN ) }}

This constructor performs  the contruction of a object through given a minimal set of data.  It is composed of the following operations :

\begin{verbatim}
Object::object(AttributeType1 data1,...,AttributeTypeN dataN )
   {
   Object(); // Default constructor

   // Loading of Simple type attributes
   this->attribute1 = data1 ; //loading of the attributes
   ....
   this->attributeN = dataN ;

   //Object attributes

   // if data are present for this object call the specific constructor
   attributeObject1 = new AttributeObject1type(data11, data 1N);

   // if not, do we need to call the default constructor (new needed ?)
   attributeObject1 = new AttributeObject1type();
   // XML Management, node creation, and link downwards ? see NDR below
   }
\end{verbatim} 

 \begin{ndr}
   \begin{enumerate}
   \item Define the procedure for the XML management :
     \begin{itemize}
     \item  Where  is the constructor of the assciated ObjectXML ? 
     \item  Use of the SaveObjectXML ?
     \item  Creation of XML sub -tree ? Advantages : Don't need to know what is the father node to fullfill a node at the moment of construction
     \end{itemize}
   \item Define if the object in attributes must created now of after and then link
   \end{enumerate}
 \end{ndr}


\subsection{Constructor from the ObjectXML of the SiconosModelXML tree {\small \tt Object::object(ObjectXML objectxml) }}

This constructor performs  the contruction of a object given a XML node. there is two steps in this type of contruction  based on the two-associted objetcs, i.e, Object, ObjectXML (for instance, DynamicalSystem and DynamicalSystemXML). The first one is the loading of the ObjectXML from the XML node.Tis step will be detailed in the section \ref{Sec:XMLnode}. The second  step is the loading of the attributes of the object from the ObjectXML.


 It is composed of the following operations :

\begin{verbatim}
Object::Object(ObjectXML objectxml)
   {
    Object(); // Default constructor
    this->objectxml = objectXML;        
    this->fillObjectWithDSXML(); // Loading of the attribute from the ObjectXML
    this->linkObjectXML();       // Link downwards
   }
\end{verbatim} 



 \begin{ndr}
   \begin{enumerate}
   \item Define the procedure for the XML management of the child :
     \begin{itemize}
     \item  Where  is the constructor of the childXML ?
     \end{itemize}
   \item Perhaps rename the method fill and link
   \end{enumerate}
 \end{ndr}



\clearpage