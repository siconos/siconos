This chapter provides guidelines and rules for developpers to implement new classes. 

\section{Constructors}
For each object three (at least) constructors are defined: 
\begin{itemize}
\item default.\\
This constructor is declared as a private or protected member function, making it accessible only to base and derived classes functions. 
\item from an xml file:\\
  to perform  the contruction of an object given by an XML node. It consists in two steps based on the two-associated objetcs, i.e, Object and ObjectXML (for instance, DynamicalSystem and DynamicalSystemXML). The first one is a loading of the ObjectXML from the XML node.This step will be detailed in the section \ref{Sec:XMLnode}. The second  step is the loading of the attributes of the object from the ObjectXML. The operations are described in the Listing~\ref{ConstructorObjectXML}.
\textbf{\textit{Remark :}} The SiconosModel owns the only constructor that needs a string corresponding to an XML file, instead of an XML object.\\
\textbf{\textit{Remark :}} About the DSInputOutput. In the XML file, they are defined a the same level than the dynamical system, whereas they are linked to these dynamical systems. This position is due to the fact one DSInputOutput can be used by several dynamical systems. So, during the loading of the NSDSXML,  DSIOXML objects have to be loaded before DSXML objects to be able to link the DSXML with their DSIOXML.\\
 \begin{ndr}
   \begin{enumerate}
   \item Define the procedure for the XML management of the child :
     \begin{itemize}
     \item The constructor for the Child Object (Externalobject) is in the link ?
     \end{itemize}
   \item Perhaps rename the method fill and link, or include it directly in the constructor ?
   \end{enumerate}
 \end{ndr}
\item from a minimum set of data
\end{itemize}
Some rules ...\\
for base class: 
\begin{itemize}
\item all members should be initialized in the constructor, as much as possible in the constructor list (see example below) 
\item Set all the pointers to 0.
\item Memory allocation for all pointers only when the size is known
\end{itemize}
Example:\\
%
% This file was automatically produced at Nov  5 1998, 12:43:56 by
% c++2latex constructor_list.cpp
%
\expandafter\ifx\csname indentation\endcsname\relax%
\newlength{\indentation}\fi
\setlength{\indentation}{0.5em}
\begin{flushleft}
\mbox{}\\
{$//$\it{} The constructor list for members initialisation ...{}\mbox{}\\
}BaseClass::BaseClass():  objectMemberA(initValue), objectMemberB(initValue), size(valSize), objectPtr(0), $\ldots$ \mbox{}\\
\{\mbox{}\\
$\ldots$\mbox{}\\
{$//$\it{} Memory allocation{}\mbox{}\\
}\hspace*{1\indentation}objectPtr = {\bf new} TypeObjectPtr(size); {$//$\it{} ok {}\mbox{}\\
}\hspace*{1\indentation}\mbox{}\\
{$//$\it{}  objectPtr $=$ new TypeObjectPtr(); // no!  {}\mbox{}\\
}$\ldots$\mbox{}\\
\}\mbox{}\\
\mbox{}\\
\mbox{}\\
\end{flushleft}

for derived class: 
\begin{itemize}
\item same rules as base class.
\item Explicit call in the constructor list to one of the base class constructors (not necessary the default one). 
\end{itemize}
\begin{ndr}
  Wouldn't it be usefull to explicitly define a copy constructor?
\end{ndr}


\section{Destructor}
For each class, a destructor is explicitly defined, as a virtual function when derived classes exist. This can be understood by reading the following lines (for more details see \cite{Eckel2000}, chapter 15 p665): 
%
% This file was automatically produced at Nov  5 1998, 12:43:56 by
% c++2latex virtual_destructor.cpp
%
\expandafter\ifx\csname indentation\endcsname\relax%
\newlength{\indentation}\fi
\setlength{\indentation}{0.5em}
\begin{flushleft}
{\bf int} main() \{\mbox{}\\
\mbox{}\\
\hspace*{2\indentation}{$//$\it{} Derived1 a class derived from Base1, a class with no virtual destructor.{}\mbox{}\\
}\hspace*{2\indentation}Base1 $\ast$bp = {\bf new} Derived1; \mbox{}\\
\hspace*{2\indentation}{\bf delete} bp; {$//$\it{} call only the destructor of Base1 $=$$>$ potential bug {}\mbox{}\\
}\mbox{}\\
\hspace*{2\indentation}{$//$\it{} Derived2 a class derived from Base2, a class with a virtual destructor.{}\mbox{}\\
}\hspace*{2\indentation}Base2 $\ast$bp2 = {\bf new} Derived2; \mbox{}\\
\hspace*{2\indentation}{\bf delete} bp2; {$//$\it{} call the destructor of Derived2 and then the one of Base2{}\mbox{}\\
}\hspace*{1\indentation}\}\mbox{}\\
\end{flushleft}

In this destructor each pointer member should be deleted (if allocated with new in constructor) and set to 0. 
%\section{Pointer members}
\section{Functions members}
\begin{itemize}
\item According to \cite{Eckel2000} (see details in chapter 8 p351 and chapter 11 p451), the first choice when passing an argument is to pass by const reference:\\
reference $\Rightarrow$ avoid pass by value, which means copy of data by creation of a temporary variable (always const). \\
const $\Rightarrow$ input argument can be a const and particularly a return value of another function. \\
This should be clear by reading the following example extracted from  \cite{Eckel2000} p 352: 
%
% This file was automatically produced at Nov  5 1998, 12:43:56 by
% c++2latex example_ref1.cpp
%
\expandafter\ifx\csname indentation\endcsname\relax%
\newlength{\indentation}\fi
\setlength{\indentation}{0.5em}
\begin{flushleft}
{\bf class} X ();\mbox{}\\
\mbox{}\\
X f() \{{\bf return} X(); \} {$//$\it{} return by value {}\mbox{}\\
}\mbox{}\\
{\bf void} g1(X\&) \{\} {$//$\it{} Pass by non-const reference {}\mbox{}\\
}\mbox{}\\
{\bf void} g2({\bf const} X\&) \{\} {$//$\it{} Pass by const reference{}\mbox{}\\
}\mbox{}\\
{\bf int} main()\mbox{}\\
\{\mbox{}\\
\hspace*{2\indentation}{$//$\it{} g1(f()); Error: const temporary created by f(){}\mbox{}\\
}\hspace*{2\indentation}g2(f()); {$//$\it{} ok {}\mbox{}\\
}\}\mbox{}\\
\end{flushleft}

\item Any member function that do not modify members should be declared as a const, this to allow const declaration for the class object. 
\end{itemize}

\section{Getters and setters}
Since object members are private data, getters and setters functions should be implemented. 
They should respect the following rules:
\begin{itemize}
\item be implemented as inline functions in the header file
\item name ended by ``Ptr'' when function handle pointers
\item respect previous remarks about function members 
\end{itemize}
For ``classical'' type (ie non pointer) object members we set two functions, one to get the value without any possibility
 to change it (ie const attribute) and the other to set the value. A third one may possibly be added, to return the adress of the member. \\
For pointer type members, four functions are implemented: 
\begin{itemize}
\item 2 to get/set the pointer
\item 2 to get/set the value of the pointed object
\end{itemize}
Finally, the corresponding header file looks like: \\
%
% This file was automatically produced at Nov  5 1998, 12:43:56 by
% c++2latex getter_and_setter.h
%
\expandafter\ifx\csname indentation\endcsname\relax%
\newlength{\indentation}\fi
\setlength{\indentation}{0.5em}
\begin{flushleft}
\mbox{}\\
$\ldots$ \mbox{}\\
\mbox{}\\
{\bf public}:\mbox{}\\
\mbox{}\\
{$//$\it{} --- GETTERS AND SETTERS ---{}\mbox{}\\
}\mbox{}\\
{$//$\it{} -- for non pointers members --{}\mbox{}\\
}{$//$\it{} get objectMemberA{}\mbox{}\\
}{\bf inline} {\bf const} typeObject getObjectMemberA() {\bf const} \{ {\bf return} objectMemberA; \}\mbox{}\\
\hspace*{8\indentation}\mbox{}\\
{$//$\it{} assign newValue to objectMemberA {}\mbox{}\\
}{\bf inline} {\bf void} setObjectMemberA({\bf const} typeObject\& newValue) \{ objectMemberA = newValue; \}\mbox{}\\
\mbox{}\\
{$//$\it{} return a pointer on objectMemberA (optional){}\mbox{}\\
}{\bf inline} {\bf const} typeObject$\ast$ getObjectMemberAPtr() {\bf const} \{ {\bf return} \&objectMemberA; \}\mbox{}\\
\mbox{}\\
{$//$\it{} -- for pointer type members --{}\mbox{}\\
}\mbox{}\\
{$//$\it{} a - get the value of pointed object {}\mbox{}\\
}{\bf inline} {\bf const} typeObject getObjectMemberB() {\bf const} \{ {\bf return} $\ast$objectMemberB; \}\mbox{}\\
\mbox{}\\
{$//$\it{} b - get the pointer {}\mbox{}\\
}{\bf inline} typeObject$\ast$ getObjectMemberBPtr() {\bf const} \{ {\bf return} objectMemberB; \}\mbox{}\\
\mbox{}\\
{$//$\it{} c - set the value of the pointed object{}\mbox{}\\
}{\bf inline} {\bf void} setObjectMemberB({\bf const} typeObject\& newValue) \{ $\ast$objectMemberB = newValue; \}\mbox{}\\
\mbox{}\\
{$//$\it{} d - set the pointer {}\mbox{}\\
}{\bf inline} {\bf void} setObjectMemberBPtr(typeObject$\ast$ newPtr) \{ {\bf delete} objectMemberB; objectMemberB=0; objectMemberB = newPtr; \}\mbox{}\\
\hspace*{8\indentation}\mbox{}\\
{$//$\it{} --- PRIVATE/PROTECTED MEMBERS ---{}\mbox{}\\
}\mbox{}\\
{\bf protected}:\mbox{}\\
\mbox{}\\
typeObject ObjectMemberA ;\mbox{}\\
\mbox{}\\
typeObject $\ast$ObjectMemberB;\mbox{}\\
\mbox{}\\
\end{flushleft}

Some remarks: 
\begin{itemize}
\item the get function (a), return a const, that means it can not be an lvalue, and so we can not write
getObjectMemberB = ...
\item (a) and (b) are const, that ensure they do not modify data members.
\item input argument in (c) is const and so can not be modified. 
\item a call to (d) means that any change on objectMemberB implies change on newPtr. 
\item for (d) function, it is necessary to first delete the data member and then to reassign it, this to avoid a ``double'' new (one on newPtr and another on objectMemberB), leading to error when call to delete at the end. 
\end{itemize}


\section{Summary and reference file}
As an example, the preceding remarks are summarized in the following header and cpp files\\
Header file: \\
%
% This file was automatically produced at Nov  5 1998, 12:43:56 by
% c++2latex reference_file_h.h
%
\expandafter\ifx\csname indentation\endcsname\relax%
\newlength{\indentation}\fi
\setlength{\indentation}{0.5em}
\begin{flushleft}
{$//$\it{}$=$$=$$=$$=$$=$$=$$=$$=$$=$$=$$=$$=$$=$$=$$=$ BASE CLASS $=$$=$$=$$=$$=$$=$$=$$=$$=$$=$$=$$=$$=$$=$$=$$=$$=$$=$$=$$=$$=$$=$$=$$=$$=$$=$$=$$=$$=${}\mbox{}\\
}{\tt \#ifndef} CLASS\_NAME\_H\mbox{}\\
{\tt \#define} CLASS\_NAME\_H\mbox{}\\
\mbox{}\\
{$//$\it{} include ...{}\mbox{}\\
}\mbox{}\\
{\bf class} BaseClassName\mbox{}\\
\mbox{}\\
\{\mbox{}\\
\hspace*{1\indentation}{\bf public}:\mbox{}\\
\hspace*{2\indentation}\mbox{}\\
\hspace*{2\indentation}{$//$\it{} --- CONSTRUCTORS ---{}\mbox{}\\
}\hspace*{2\indentation}\mbox{}\\
\hspace*{2\indentation}{$//$\it{} From XML {}\mbox{}\\
}\hspace*{2\indentation}BaseClass::BaseClass(DSXML $\ast$ dsXML);\mbox{}\\
\hspace*{2\indentation}\mbox{}\\
\hspace*{2\indentation}{$//$\it{} From a minimum set of data{}\mbox{}\\
}\hspace*{2\indentation}BaseClass::BaseClass(typeObject A, typeObject$\ast$ B);\mbox{}\\
\mbox{}\\
\hspace*{2\indentation}{$//$\it{} --- DESTRUCTOR --- {}\mbox{}\\
}\hspace*{2\indentation}{\bf virtual} BaseClass::$\sim$BaseClass();\mbox{}\\
\mbox{}\\
\hspace*{2\indentation}{$//$\it{} --- GETTERS AND SETTERS ---{}\mbox{}\\
}\hspace*{2\indentation}\mbox{}\\
\hspace*{2\indentation}{$//$\it{} get sizeB{}\mbox{}\\
}\hspace*{2\indentation}{\bf inline} {\bf const} {\bf int} getSizeB() {\bf const} \{ {\bf return} sizeB;\}\mbox{}\\
\hspace*{8\indentation}\mbox{}\\
\hspace*{2\indentation}{$//$\it{} assign newValue to sizeB{}\mbox{}\\
}\hspace*{2\indentation}{\bf inline} {\bf void} setSizeB({\bf const} {\bf int}\& newValue) \{ sizeB = newValue; \}\mbox{}\\
\mbox{}\\
\hspace*{2\indentation}{$//$\it{} get the value of objectMemberB{}\mbox{}\\
}\hspace*{2\indentation}{\bf inline} {\bf const} Type getObjectMemberB() {\bf const} \{ {\bf return} $\ast$objectMemberB; \}\mbox{}\\
\mbox{}\\
\hspace*{2\indentation}{$//$\it{} get the pointer {}\mbox{}\\
}\hspace*{2\indentation}{\bf inline} Type$\ast$ getObjectMemberBPtr() {\bf const} \{ {\bf return} objectMemberB; \}\mbox{}\\
\mbox{}\\
\hspace*{2\indentation}{$//$\it{} set the value of the pointed object{}\mbox{}\\
}\hspace*{2\indentation}{\bf inline} {\bf void} setObjectMemberB({\bf const} Type\& newValue) \{ $\ast$objectMemberB = newValue; \}\mbox{}\\
\mbox{}\\
\hspace*{2\indentation}{$//$\it{} set the pointer {}\mbox{}\\
}\hspace*{2\indentation}{\bf inline} {\bf void} setObjectMemberBPtr(Type$\ast$ newPtr) \{ {\bf delete} objectMemberB; objectMemberB=0; objectMemberB = newPtr; \}\mbox{}\\
\hspace*{2\indentation}{$//$\it{} ...{}\mbox{}\\
}\hspace*{2\indentation}\mbox{}\\
\hspace*{2\indentation}{$//$\it{} --- OTHER FUNCTIONS --- {}\mbox{}\\
}\hspace*{2\indentation}{$//$\it{} able to change data members{}\mbox{}\\
}\hspace*{2\indentation}returnType func1($\ldots$) ;\mbox{}\\
\hspace*{2\indentation}\mbox{}\\
\hspace*{2\indentation}{$//$\it{} which is not supposed to modify data members{}\mbox{}\\
}\hspace*{2\indentation}returnType func2($\ldots$) {\bf const} ;\mbox{}\\
\hspace*{2\indentation}\mbox{}\\
\hspace*{2\indentation}{$//$\it{} which can not modify input arguments{}\mbox{}\\
}\hspace*{2\indentation}returnType func3({\bf const} typeInput\& inputArg); \mbox{}\\
\mbox{}\\
\hspace*{2\indentation}{$//$\it{} with return argument that can not be modified $=$$>$ can not be set as an lValue{}\mbox{}\\
}\hspace*{2\indentation}{\bf const} returnType func4($\ldots$); \mbox{}\\
\mbox{}\\
\hspace*{2\indentation}{$//$\it{} and combinations of these previous cases ...{}\mbox{}\\
}\hspace*{4\indentation}\mbox{}\\
\hspace*{2\indentation}{$//$\it{} --- PRIVATE/PROTECTED MEMBERS ---{}\mbox{}\\
}\mbox{}\\
\hspace*{1\indentation}{\bf protected}:\mbox{}\\
\mbox{}\\
\hspace*{2\indentation}{$//$\it{} -- Default constructor --{}\mbox{}\\
}\hspace*{2\indentation}BaseClass::BaseClass();\mbox{}\\
\mbox{}\\
\hspace*{2\indentation}{$//$\it{} -- data members --{}\mbox{}\\
}\hspace*{2\indentation}typeObject ObjectMemberA ;\mbox{}\\
\mbox{}\\
\hspace*{2\indentation}typeObject $\ast$ObjectMemberB;\mbox{}\\
\mbox{}\\
\hspace*{2\indentation}{\bf int} sizeB; \mbox{}\\
\hspace*{2\indentation}\mbox{}\\
\hspace*{2\indentation}{$//$\it{} ...{}\mbox{}\\
}\mbox{}\\
\};\mbox{}\\
\mbox{}\\
{\tt \#endif} \mbox{}\\
\mbox{}\\
{$//$\it{}$=$$=$$=$$=$$=$$=$$=$$=$$=$$=$$=$$=$$=$$=$$=$ DERIVED CLASS $=$$=$$=$$=$$=$$=$$=$$=$$=$$=$$=$$=$$=$$=$$=$$=$$=$$=$$=$$=$$=$$=$$=$$=$$=$$=$$=$$=$$=${}\mbox{}\\
}\mbox{}\\
{$//$\it{} warning !! Implemented in a different file !!{}\mbox{}\\
}\mbox{}\\
{\tt \#ifndef} DERIVED\_CLASS\_NAME\_H\mbox{}\\
{\tt \#define} DERIVED\_CLASS\_NAME\_H\mbox{}\\
\mbox{}\\
{$//$\it{} include ...{}\mbox{}\\
}\mbox{}\\
{\bf class} DerivedClassName: {\bf public} BaseClassName\mbox{}\\
\{\mbox{}\\
\hspace*{1\indentation}{\bf public}:\mbox{}\\
\hspace*{2\indentation}\mbox{}\\
\hspace*{2\indentation}{$//$\it{} --- CONSTRUCTORS ---{}\mbox{}\\
}\hspace*{2\indentation}{$//$\it{} From XML {}\mbox{}\\
}\hspace*{2\indentation}DerivedClass::DerivedClass(DSXML $\ast$ dsXML);\mbox{}\\
\hspace*{2\indentation}{$//$\it{} From a minimum set of data{}\mbox{}\\
}\hspace*{2\indentation}DerivedClass::DerivedClass(typeObject A, typeObject$\ast$ B);\mbox{}\\
\hspace*{2\indentation}{$//$\it{} --- DESTRUCTOR --- {}\mbox{}\\
}\hspace*{2\indentation}DerivedClass::$\sim$DerivedClass();\mbox{}\\
\mbox{}\\
\hspace*{2\indentation}{$//$\it{} --- GETTERS AND SETTERS ---{}\mbox{}\\
}\hspace*{2\indentation}{$//$\it{} ...{}\mbox{}\\
}\mbox{}\\
\hspace*{2\indentation}{$//$\it{} --- OTHER FUNCTIONS --- {}\mbox{}\\
}\hspace*{2\indentation}{$//$\it{} ...{}\mbox{}\\
}\hspace*{2\indentation}\mbox{}\\
\hspace*{2\indentation}{$//$\it{} --- PRIVATE/PROTECTED MEMBERS ---{}\mbox{}\\
}\hspace*{2\indentation}\mbox{}\\
\hspace*{1\indentation}{\bf protected}:\mbox{}\\
\mbox{}\\
\hspace*{2\indentation}{$//$\it{} -- Default constructor --{}\mbox{}\\
}\hspace*{2\indentation}DerivedClass::DerivedClass();\mbox{}\\
\mbox{}\\
\hspace*{2\indentation}{$//$\it{} -- data members --{}\mbox{}\\
}\hspace*{2\indentation}typeObject $\ast$ObjectMemberC ;\mbox{}\\
\mbox{}\\
\hspace*{2\indentation}{\bf int} sizeC; \mbox{}\\
\hspace*{2\indentation}\mbox{}\\
\hspace*{2\indentation}{$//$\it{} ...{}\mbox{}\\
}\mbox{}\\
\};\mbox{}\\
\mbox{}\\
{\tt \#endif} \mbox{}\\
\end{flushleft}

cpp file: \\
%
% This file was automatically produced at Nov  5 1998, 12:43:56 by
% c++2latex reference_file_cpp.cpp
%
\expandafter\ifx\csname indentation\endcsname\relax%
\newlength{\indentation}\fi
\setlength{\indentation}{0.5em}
\begin{flushleft}
{$//$\it{}$=$$=$$=$$=$$=$$=$$=$$=$$=$$=$$=$$=$$=$$=$$=$ BASE CLASS $=$$=$$=$$=$$=$$=$$=$$=$$=$$=$$=$$=$$=$$=$$=$$=$$=$$=$$=$$=$$=$$=$$=$$=$$=$$=$$=$$=$$=${}\mbox{}\\
}\mbox{}\\
{$//$\it{} include ...{}\mbox{}\\
}{$//$\it{} --- CONSTRUCTORS ---{}\mbox{}\\
}\mbox{}\\
{$//$\it{} From XML {}\mbox{}\\
}BaseClass::BaseClass(DSXML $\ast$ dsXML): objectMemberA(initValue), sizeB(0), objectMemberB(0), $\ldots$ \mbox{}\\
\{\mbox{}\\
\hspace*{2\indentation}{$//$\it{} xml data loading {}\mbox{}\\
}\hspace*{2\indentation}sizeB = $\ldots$;\mbox{}\\
\hspace*{2\indentation}{$//$\it{} ...{}\mbox{}\\
}\hspace*{2\indentation}{$//$\it{} Memory allocation{}\mbox{}\\
}\hspace*{2\indentation}objectMemberB = {\bf new} typeObject(sizeB) ;\mbox{}\\
\} \mbox{}\\
\hspace*{1\indentation}\mbox{}\\
{$//$\it{} From a minimum set of data{}\mbox{}\\
}BaseClass::BaseClass(typeObject A, typeObject$\ast$ B, {\bf int} size):objectMemberA(A), sizeB(size), objectMemberB(0), $\ldots$ \mbox{}\\
\{\mbox{}\\
\hspace*{2\indentation}{$//$\it{} Memory allocation{}\mbox{}\\
}\hspace*{2\indentation}objectMemberB = {\bf new} typeObject(sizeB) ;\mbox{}\\
\hspace*{2\indentation}{$//$\it{} Assign B value to objectMemberB{}\mbox{}\\
}\hspace*{2\indentation}{$//$\it{} warning! Do not set objectMemberB $=$ B: should cause segmentation error when call to destructor if B has been declared with a new. {}\mbox{}\\
}\hspace*{2\indentation}$\ast$objectMemberB = $\ast$B ;\mbox{}\\
\hspace*{2\indentation}{$//$\it{} ...{}\mbox{}\\
}\}\mbox{}\\
\mbox{}\\
{$//$\it{} --- DESTRUCTOR --- {}\mbox{}\\
}BaseClass::$\sim$BaseClass()\mbox{}\\
\{\mbox{}\\
\hspace*{2\indentation}{\bf delete} objectMemberB;\mbox{}\\
\hspace*{2\indentation}objectMemberB=0; \mbox{}\\
\}\mbox{}\\
\mbox{}\\
{$//$\it{} --- Default constructor ---{}\mbox{}\\
}BaseClass::BaseClass(): objectMemberA(0), sizeB(0), objectMemberB(0), $\ldots$ \mbox{}\\
\{\}\mbox{}\\
\mbox{}\\
\mbox{}\\
{$//$\it{}$=$$=$$=$$=$$=$$=$$=$$=$$=$$=$$=$$=$$=$$=$$=$ DERIVED CLASS $=$$=$$=$$=$$=$$=$$=$$=$$=$$=$$=$$=$$=$$=$$=$$=$$=$$=$$=$$=$$=$$=$$=$$=$$=$$=$$=$$=$$=${}\mbox{}\\
}\mbox{}\\
{$//$\it{} warning !! Implemented in a different file !!{}\mbox{}\\
}{$//$\it{} include ...{}\mbox{}\\
}{$//$\it{} --- CONSTRUCTORS ---{}\mbox{}\\
}\mbox{}\\
{$//$\it{} From XML {}\mbox{}\\
}DerivedClass::DerivedClass(DSXML $\ast$ dsXML): BaseClass(dsXML), objectMemberC(0),sizeC(0) $\ldots$\mbox{}\\
\{\mbox{}\\
\hspace*{2\indentation}{$//$\it{} xml data loading {}\mbox{}\\
}\hspace*{2\indentation}sizeC= $\ldots$;\mbox{}\\
\hspace*{2\indentation}{$//$\it{} ...{}\mbox{}\\
}\hspace*{2\indentation}{$//$\it{} Memory allocation{}\mbox{}\\
}\hspace*{2\indentation}objectMemberC = {\bf new} typeObject(sizeC) ;\mbox{}\\
\} \mbox{}\\
\hspace*{1\indentation}\mbox{}\\
{$//$\it{} From a minimum set of data{}\mbox{}\\
}DerivedClass::DerivedClass(typeObject$\ast$ C, typeObject A, typeObject$\ast$ B, {\bf int} size): BaseClass(A,B,size), objectMemberC(C), sizeC(size) $\ldots$\mbox{}\\
\{\mbox{}\\
\hspace*{2\indentation}objectMemberC = {\bf new} typeObject(sizeC) ;\mbox{}\\
\hspace*{2\indentation}$\ast$objectMemberC = $\ast$C ;\mbox{}\\
\hspace*{2\indentation}{$//$\it{} ...{}\mbox{}\\
}\}\mbox{}\\
\mbox{}\\
{$//$\it{} --- DESTRUCTOR --- {}\mbox{}\\
}DerivedClass::$\sim$DerivedClass()\mbox{}\\
\{\mbox{}\\
\hspace*{2\indentation}{\bf delete} objectMemberC;\mbox{}\\
\hspace*{2\indentation}objectMemberC=0;\mbox{}\\
\hspace*{2\indentation}{$//$\it{} Warning: do not delete baseClass members, the base destructor is automatically called.{}\mbox{}\\
}\}\mbox{}\\
\mbox{}\\
{$//$\it{} ...{}\mbox{}\\
}\mbox{}\\
{$//$\it{} --- Default constructor ---{}\mbox{}\\
}DerivedClass::DerivedClass(): BaseClass(), objectMemberC(0),sizeC(0)$\ldots$\mbox{}\\
\{\}\mbox{}\\
\mbox{}\\
\end{flushleft}



