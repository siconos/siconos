%---------------------------------------------------------------------------------------%
%					section						%
%---------------------------------------------------------------------------------------%

This chapter reports the development of the Python Interface for the Kernel.

\section{C++ - Python wrapper choice}

There mainly exists two free tools to wrap C++ code with python : Swig and Boost. \\ 

%---------------	sub-section		----------------------------------------------------%

\subsection{Swig}

SWIG is a software development tool that connects programs written in C and C++ with a variety of high-level programming languages. SWIG is used with different types of languages including common scripting languages such as Perl, Python, Tcl/Tk and Ruby (http://www.swig.org). 

The main task to connect C++ with Python using SWIG is to write an interface file. It references the header files that must be wrap and some types declarations for the data strctures that cannot be directly and automatically wrapped.

SWIG is really easy to use, but does not support comnpletely every features of advanced C++ :
\begin{itemize}
\item friend functions
\item overloaded operators with the same number of parameters and close types (int / double, etc.).
\end{itemize}

\subsection{Boost}

Boost is a complete framework for C++ which provides advanced features and completes the Standard Template Library. Its Python wrapper is only one functionnality among many others.

for the developement of the Kernel Python interface, we have chosen SWIG because this tool is easier to use and using Boost just for wrapping C++ with Python seemed not an appropriate solution.


\section{Kernel wrapping}

Connecting the Kernel with Python was pretty fast. We only encountered a problem with down-casting operations. In the kernel, we have STL containers of pointers. For example, an NSDS stores the dynamical systems, but does not know their specific types. So we can access to the specific data of a type of dynamical system after a down-casting operation (static\_cast and dynamic\_cast in C++). This operation is not allowed in Python. To solve this problem, we wrote a cast method in each class which is not a base class. This method wraps the C++ operator of dynamic casting and returns a pointer on the class.\\

The connection of STL containers is not completely automatic. We have to define a Python type for each data type stored in such a container. This type definition must be placed in the interface file. For example, to be able to use a container of pointers on Dynamical systems, we add this command in the Swig interface file: 
\begin{verbatim}
%include "std_vector.i";
namespace std {
    %template(dsVector) vector<DynamicalSystem*>;
}
\end{verbatim}



\section{Miscelleanous}

In the last versions of Swig (1.3.23, 1.3.24), a bug makes fail the compilation of the interface. This is a simple syntax error and we wrote a bash script to correct this. This bug should be fixed soon in Swig distributions.
