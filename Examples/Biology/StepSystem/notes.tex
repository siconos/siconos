\documentclass{article}
\usepackage{amsmath}
\title{Numerical simulation of PWL models of gene regulatory networks}
\author{V. Acary}
\begin{document}
\maketitle
\subsection*{Standard form}

\begin{equation}
  \label{eq:std}
  \left\{\begin{array}{l}
      M \dot x  = f(x,t) +  r(t) \\
      r(t) = g(x,t,\lambda) \text{ (input) }  \\
      y(t) = h(x,t,\lambda) \text{ (output) } \\
   \end{array}\right.
\end{equation}

with one of the following nonsmooth law
\begin{enumerate}
\item Sign function
  \begin{equation}
    \label{eq:sign}
    \lambda \in \mbox{Sign} (-y) \Longleftrightarrow  - y  \in N_{[-1,1]}(\lambda)
  \end{equation}
\item Step function
  \begin{equation}
    \label{eq:step}
    \lambda \in \mbox{Step} (-y) \Longleftrightarrow  - y  \in N_{[0,1]}(\lambda)
  \end{equation}
\end{enumerate}

We have 
\begin{equation}
  \label{eq:signstep}
  \begin{array}{l}
  s^+(x,\theta) = \mbox{Step}(x-\theta) = \frac 1 2 (1+ \mbox{Sign}(x-\theta))\\ \\
  s^-(x,\theta) = 1 - \mbox{Step}(x-\theta) = \frac 1 2 (1- \mbox{Sign}(x-\theta))
\end{array}
\end{equation}
\subsection*{First 2D example}
\begin{equation*}
\left\{\begin{array}{l} \dot{x}_{a} = - \gamma_{a}\, x_{a}+ \kappa_{a} \, s^{+}(x_{b},\theta_{b}^1)\, s^{-}(x_{a},\theta_{a}^2) \\
\dot{x}_{b} = - \gamma_{b}\, x_{b} + \kappa_{b} \,  s^{+}(x_{a},\theta_{a}^1)\,s^{-}(x_{b},\theta_{b}^2) \end{array}\right. .
\end{equation*}


The parameters are: $\theta_a^1=\theta_b^1=4, \theta_a^2=\theta_b^2=8, k_a=k_b=40, \gamma_a=4.5$ and $\gamma_b=1.5$.

The vector fields $f$ and $h$ can be identified as


\begin{equation}
  \label{eq:f1}
  f(x,t) = \left[
    \begin{array}{l}
      - \gamma_a x_a \\
      - \gamma_b x_b \\
    \end{array}
  \right]=\left[
    \begin{array}{l}
      - 4.5 x_a \\
      - 1.5 x_b \\
    \end{array}
  \right]
\end{equation}

\begin{equation}
  \label{eq:h1}
  h(x,t,\lambda) = -  \left[
    \begin{array}{l}
       x_a - \theta_a^1 \\      
       x_b - \theta_b^1 \\
       x_a - \theta_a^2 \\      
       x_b - \theta_b^2 \\
    \end{array}
  \right]=\left[
    \begin{array}{l}
      4- x_a \\      
      4- x_b \\
      8- x_a \\      
      8- x_b \\
    \end{array}
  \right]
\end{equation}


\subsubsection*{With the sign function}
\begin{equation}
  \label{eq:g1sign}
  \begin{array}{lcl}
  g(x,t,\lambda) &=&\displaystyle\frac 1 4  \left[
    \begin{array}{l}
      \kappa_a      (1- \mbox{sign}( x_a - \theta_a^2))((1+ \mbox{sign}( x_b - \theta_b^1)))\\
      \kappa_b      (1+ \mbox{sign}( x_a - \theta_a^1))((1- \mbox{sign}( x_b - \theta_b^2)))
    \end{array}
  \right]\\ \\
  &=& 10   \left[
    \begin{array}{l}
          (1-\lambda_3)(1+\lambda_2)\\
          (1+\lambda_1)(1-\lambda_4)
    \end{array}
  \right] 
\end{array}
\end{equation}

\subsubsection*{With the Step function}
\begin{equation}
  \label{eq:g1step}
  \begin{array}{lcl}
  g(x,t,\lambda) &=&  \left[
    \begin{array}{l}
      \kappa_a      (1- \mbox{step}( x_a - \theta_a^2))( \mbox{step}( x_b - \theta_b^1)))\\
      \kappa_b      (\mbox{step}( x_a - \theta_a^1))((1- \mbox{step}( x_b - \theta_b^2)))
    \end{array}
  \right]\\ \\
  &=& 40   \left[
    \begin{array}{l}
          \lambda_2 (1-\lambda_3)\\
          \lambda_1 (1-\lambda_4)
    \end{array}
  \right] 
\end{array}
\end{equation}

\end{document}

%%% Local Variables: 
%%% mode: latex
%%% TeX-master: t
%%% End: 
