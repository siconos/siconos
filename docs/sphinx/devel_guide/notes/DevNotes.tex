% Siconos is a program dedicated to modeling, simulation and control
% of non smooth dynamical systems.
%
% Copyright 2024 INRIA.
%
% Licensed under the Apache License, Version 2.0 (the "License");
% you may not use this file except in compliance with the License.
% You may obtain a copy of the License at
%
% http://www.apache.org/licenses/LICENSE-2.0
%
% Unless required by applicable law or agreed to in writing, software
% distributed under the License is distributed on an "AS IS" BASIS,
% WITHOUT WARRANTIES OR CONDITIONS OF ANY KIND, either express or implied.
% See the License for the specific language governing permissions and
% limitations under the License.

\documentclass[10pt]{report}
%$Id: macro.tex,v 1.10 2004/12/08 13:38:58 acary Exp $


%\usepackage{a4wide}
\textheight 25cm
\textwidth 16.5cm
\topmargin -1cm
%\evensidemargin 0cm
\oddsidemargin 0cm
\evensidemargin0cm
\usepackage{layout}


\usepackage{amsmath}
\usepackage{amssymb}
\usepackage{minitoc}
\usepackage{glosstex}
\usepackage{colortbl}
\usepackage{hhline}
\usepackage{longtable}

\usepackage{glosstex}
\def\glossaryname{Glossary of Notation}
\def\listacronymname{Acronyms}

\usepackage[outerbars]{changebar}\setcounter{changebargrey}{20}
\glxitemorderdefault{acr}{l}

%\usepackage{color}
\usepackage{graphicx,epsfig}
\graphicspath{{figure/}}
\usepackage[T1]{fontenc}
\usepackage{rotating}

%\usepackage{algorithmic}
%\usepackage{algorithm}
\usepackage{ntheorem}
\usepackage{natbib}


%\renewcommand{\baselinestretch}{2.0}
\setcounter{tocdepth}{2}     % Dans la table des matieres
\setcounter{secnumdepth}{3}  % Avec un numero.



\newtheorem{definition}{Definition}
\newtheorem{lemma}{Lemma}
\newtheorem{claim}{Claim}
\newtheorem{remark}{Remark}
\newtheorem{assumption}{Assumption}
\newtheorem{example}{Example}
\newtheorem{conjecture}{Conjecture}
\newtheorem{corollary}{Corollary}
\newtheorem{OP}{OP}
\newtheorem{problem}{Problem}
\newtheorem{theorem}{Theorem}


\newcommand{\CC}{\mbox{\rm $~\vrule height6.6pt width0.5pt depth0.25pt\!\!$C}}
\newcommand{\ZZ}{\mbox{\rm \lower0.3pt\hbox{$\angle\!\!\!$}Z}}
\newcommand{\RR}{\mbox{\rm $I\!\!R$}}
\newcommand{\NN}{\mbox{\rm $I\!\!N$}}

\newcommand{\Mnn}{\mathcal M^{n\times n}}
\newcommand{\Mnp}[2]{\ensuremath{\mathcal M^{#1\times #2}}}



\newcommand{\Frac}[2]{\displaystyle \frac{#1}{#2}}

\newcommand{\DP}[2]{\displaystyle \frac{\partial {#1}}{\partial {#2}}}

% c++ variables writting
\newcommand{\varcpp}[1]{\textit{#1}}
% itemize
\newcommand{\bei}{\begin{itemize}}
\newcommand{\ei}{\end{itemize}}

\newcommand{\ie}{i.e.}
\newcommand{\eg}{e.g.}
\newcommand{\cf}{c.f.}
\newcommand{\putidx}[1]{\index{#1}\textit{#1}}

\def\Er{{\rm I\! R}}
\def\En{{\rm I\! N}} 
\def\Ec{{\rm I\! C}}
 
\def\zc{\hat{z}}
\def\wc{\hat{w}}

\font\tete=cmr8 at 8 pt
\font\titre= cmr12 at 20 pt 
\font\titregras=cmbx12 at 20 pt

%----------------------------------------------------------------------
%                  Modification des subsubsections
%----------------------------------------------------------------------
\makeatletter
\renewcommand\thesubsubsection{\thesubsection.\@alph\c@subsubsection}
\makeatother

%----------------------------------------------------------------------
%             Redaction note environnement
%----------------------------------------------------------------------
\makeatletter
\theoremheaderfont{\scshape}
\theoremstyle{marginbreak}
\theorembodyfont{\upshape}
%\newtheorem{rque}{\bf Remarque}[chapter]
%\newtheorem{rque1}{\bf \fsc{Remarque}}[chapter] !!! \fsc est une commande french
\newtheorem{ndr1}{\textbf{\textsc{Redaction note}}}[section]

\newenvironment{ndr}%
{%
\tt
%\centerline{---oOo---}
\noindent\begin{ndr1}%
}%
{%
\begin{flushright}%
%\vspace{-1.5em}\ding{111}
\end{flushright}%
\end{ndr1}%
%\centerline{---oOo---}
}

\makeatother

%----------------------------------------------------------------------
%             Redaction note environnement V.ACARY
%----------------------------------------------------------------------
\makeatletter
\theoremheaderfont{\scshape}
\theoremstyle{marginbreak}
\theorembodyfont{\upshape}
%\newtheorem{rque}{\bf Remarque}[chapter]
%\newtheorem{rque1}{\bf \fsc{Remarque}}[chapter] !!! \fsc est une commande french
\newtheorem{ndr1va}{\textbf{\textsc{Redaction note V. ACARY}}}[section]

\newenvironment{ndrva}%
{%
\tt
%\centerline{---oOo---}
\noindent\begin{ndr1va}%
}%
{%
\begin{flushright}%
%\vspace{-1.5em}\ding{111}
\end{flushright}%
\end{ndr1va}%
%\centerline{---oOo---}
}

\makeatother
%----------------------------------------------------------------------
%             Redaction note environnement V.ACARY
%----------------------------------------------------------------------
\makeatletter
\theoremheaderfont{\scshape}
\theoremstyle{marginbreak}
\theorembodyfont{\upshape}
%\newtheorem{rque}{\bf Remarque}[chapter]
%\newtheorem{rque1}{\bf \fsc{Remarque}}[chapter] !!! \fsc est une commande french
\newtheorem{ndr1fp}{\textbf{\textsc{Redaction note F. PERIGNON}}}[section]

\newenvironment{ndrfp}%
{%
\tt
%\centerline{---oOo---}
\noindent\begin{ndr1fp}%
}%
{%
\begin{flushright}%
%\vspace{-1.5em}\ding{111}
\end{flushright}%
\end{ndr1fp}%
%\centerline{---oOo---}
}

\makeatother
%----------------------------------------------------------------------
%                  Chapter head enviroment
%----------------------------------------------------------------------
\newenvironment{chapter_head}
{%
\begin{center}%
-------------------- oOo --------------------\\%
\ \\%
\begin{minipage}[]{14cm}%
\noindent\normalsize\advance\baselineskip-1pt %
}%
{%
\par\end{minipage}%
\ \\%
\ \\%
-------------------- oOo --------------------
\end{center}%
\vspace*{\stretch{1}}%
\clearpage%
\thispagestyle{empty}%
\vspace*{\stretch{1}}%
\minitoc%
\vspace*{\stretch{2}}%
\clearpage%
}

%%% Local Variables: 
%%% mode: latex
%%% TeX-master: "report"
%%% End: 

\usepackage{graphicx, amsmath, amsfonts, amssymb, mathtools}
\usepackage{psfrag}
\usepackage{fancyhdr}
\usepackage{subfigure}
\usepackage{cases}
\usepackage{esvect}


\usepackage{placeins}

%\renewcommand{\baselinestretch}{1.2}
\textheight 23cm
\textwidth 16cm
\topmargin 0cm
%\evensidemargin 0cm
\oddsidemargin 0cm
\evensidemargin 0cm
\usepackage{layout}
\usepackage{mathpple}
\usepackage[T1]{fontenc}

%\usepackage{array}
\makeatletter
\renewcommand\bibsection{\paragraph{References
     \@mkboth{\MakeUppercase{\bibname}}{\MakeUppercase{\bibname}}}}
\makeatother
\usepackage{lastpage}
\usepackage{showlabels}

\setlength{\fboxrule}{2pt}
\setlength{\fboxsep}{4mm}
%% style des entetes et des pieds de page
\fancyhf{} % nettoie le entetes et les pieds
\fancyhead[L]{\texttt{Siconos Development team --   Notes }}
\fancyhead[C]{}
\fancyhead[R]{\texttt{\thepage/\pageref{LastPage}}}
\fancyfoot[L]{}
\fancyfoot[C]{}
\fancyfoot[R]{\texttt{file DevNotes.tex -- \isodayandtime}}


\everymath{\displaystyle}
\begingroup
\count0=\time \divide\count0by60 % Hour
\count2=\count0 \multiply\count2by-60 \advance\count2by\time
% Min
\def\2#1{\ifnum#1<10 0\fi\the#1}
\xdef\isodayandtime{\the\year-\2\month-\2\day\space\2{\count0}:%
\2{\count2}}
\endgroup

\graphicspath{{./Figures}}


\def\free{{\sf free}}

%\usepackage{xcolor} % for colors in document (red for title, gray for English)
% Hyperref: links in document. Most links are hidden (black color) except URLs (blue)
%\usepackage[colorlinks=true,urlcolor=blue,linkcolor=black,citecolor=black]{hyperref}

\begin{document}
\thispagestyle{empty}
\title{Developer's Notes}
\author{Siconos Development Team}

\date{\today}
\maketitle

\tableofcontents
\clearpage
\pagestyle{fancy}

\chapter{OneStepNSProblem formalisation for several interactions}

\begin{table}[!ht]
  \begin{tabular}{|l|l|}
    \hline
    author  & F. P\'erignon \\
    \hline
    date    & May 16, 2006 \\ 
    \hline
    version & ? \\
    \hline
  \end{tabular}
\end{table}



\section{LinearDS - Linear Time Invariant Relations}
\subsection{General notations}
We consider $n$ dynamical systems of the form:
\begin{equation}
\dot x_i = A_i x_i + R_i 
\end{equation}
Each system if of dimension $n_i$, and we denote $N = \displaystyle{\sum_{i=1}^{n} n_i}$. \\
An interaction, $I_{\alpha}$ is composed with a non smooth law, $nslaw_{\alpha}$ and a relation:
\begin{equation}
y_{\alpha} = C_{\alpha}X_{\alpha} + D_{\alpha}\lambda_{\alpha}
\end{equation}
The ``dimension'' of the interaction, ie the size of vector $y_{\alpha}$, is denoted $m_{\alpha}$ and we set: 
$$ M = \sum_{\alpha=1}^{m} m_{\alpha}$$
$m$ being the number of interactions in the Non Smooth Dynamical System.  \\
$X_{\alpha}$ is a vector that represents the DS concerned by the interaction. Its dimension is noted $N_{\alpha}$, this for $n_{\alpha}$ systems in the interaction. \\
$C_{\alpha}$ is a $m_{\alpha} \times N_{\alpha}$ row-blocks matrix and $D_{\alpha}$ a $m_{\alpha} \times m_{\alpha}$ square matrix. \\
\begin{equation}
C_{\alpha}=\left[\begin{array}{ccc} 
C_{\alpha}^i & C_{\alpha}^j & ...\end{array}\right]
\end{equation}
with $i,j,...\in \mathcal{DS}_{\alpha}$ which is the set of DS belonging to interaction $\alpha$.\\
We also have the following relation: 
\begin{equation}
\left[\begin{array}{c} 
R_{\alpha}^i \\
R_{\alpha}^j \\
...  
\end{array}\right] = B_{\alpha}\lambda_{\alpha}
=\left[\begin{array}{c} 
B_{\alpha}^i \\
B_{\alpha}^j \\
...
\end{array}\right]\lambda_{\alpha}
\end{equation}
$R_{\alpha}^i$ represents the contribution of interaction $\alpha$ on the reaction of the dynamical system $i$, and $B_{\alpha}^i$ is a $n_i \times m_{\alpha}$ block matrix. \\ 
And so: 
\begin{equation}
R_i = \sum_{\beta\in\mathcal{I}_i}R_{\beta}^i=\sum_{\beta\in\mathcal{I}_i}B^i_{\beta} \lambda_{\beta}
\end{equation}
with $\mathcal{I}_i$ the set of interactions in which dynamical system number $i$ is involved. \\
Introducing the time discretization, we get: 
\begin{eqnarray}
x_i^{k+1}-x_i^k = h A_i x_i^{k+1} + h R_i^{k+1}  \\
\nonumber\\
y_{\alpha}^{k+1} = C_{\alpha}X_{\alpha}^{k+1} + D_{\alpha}\lambda_{\alpha}^{k+1}\\
\nonumber\\
R_i^{k+1} = \sum_{\beta\in\mathcal{I}_i}B^i_{\beta} \lambda_{\beta}^{k+1}
\end{eqnarray}
ie, with $W_i = (I-h A_i)^{-1}$: 
\begin{eqnarray}
x_i^{k+1}&=& W_i x_i^{k} + hW_i R_i^{k+1}  \\
\nonumber\\
y_{\alpha}^{k+1} &=& C_{\alpha}W_{\alpha} X_{\alpha}^{k} + C_{\alpha}hW_{\alpha}\sum_{\beta\in\mathcal{I}_i}B^i_{\beta} \lambda_{\beta}^{k+1} + D_{\alpha}\lambda_{\alpha}^{k+1} \\
&=& C_{\alpha}W_{\alpha} X_{\alpha}^{k} + (C_{\alpha}hW_{\alpha}B_{\alpha} + D_{\alpha}) \lambda_{\alpha}^{k+1} + \sum_{\beta\neq\alpha}(\sum_{i\in\mathcal{DS}_{\alpha}\cap\in\mathcal{DS}_{\beta}} hC_{\alpha}^iW_i B^i_{\beta} \lambda_{\beta}^{k+1})
\end{eqnarray}
with 
\begin{equation}\label{Walpha}
W_{\alpha}=\left[\begin{array}{ccc} 
W_i &  0   & ... \\
0   &  W_j & ...\\
0  & ... & ... \\ 
\end{array}\right]
\end{equation}
the block-diagonal matrix of all the $W$ for the dynamical systems involved in interaction $\alpha$.\\  
The global-assembled $Y$ vector, of dimension M, composed by $m$ $y_{\alpha}$ subvectors, is given by:
\begin{eqnarray}
Y_{k+1} = q_{OSNSP} + M_{OSNSP}\Lambda_{k+1}
\end{eqnarray}
or,
\begin{eqnarray}
Y_{k+1} =\left[\begin{array}{c} 
y_1 \\
...  \\
y_m
\end{array}\right]_{k+1}
&=&\left[\begin{array}{ccc} 
C_1^1 & \ldots & C_1^n \\
\vdots & \ldots & \vdots \\
C_m^1 & \ldots & C_m^n 
\end{array}\right]\left[\begin{array}{cccc} 
W_1 & 0 & \ldots &0 \\
0  & W_2 & \ddots & \vdots \\
\vdots &\ddots  & \ddots & \vdots \\
&&0& W_n
\end{array}\right]
\left[\begin{array}{c} 
x_1  \\
\vdots \\
\vdots \\
x_n 
\end{array}\right]_k \\
&+&\left[\begin{array}{cccc} 
D_1+h\sum_{j\in \mathcal{DS}_1}C_1^jW_jB_1^j & h\displaystyle{\sum_{j\in \mathcal{DS}_1\cap\mathcal{DS}_2}C_1^jW_jB_2^j} & \ldots &\\
\vdots&\ddots& &\\
& h\displaystyle{\sum_{j\in \mathcal{DS}_m}C_m^jW_jB_{m-1}^j}  & D_m+h\displaystyle{\sum_{j\in \mathcal{DS}_m\cap\mathcal{DS}_{m-1}}C_m^jW_jB_m^j} \\
\end{array}\right]\left[\begin{array}{c} 
\lambda_1  \\
\vdots \\
\lambda_m 
\end{array}\right]_{k+1} \nonumber
\end{eqnarray}
To sum it up, the block-diagonal term of matrix $M_{OSNSP}$, for block-row $\alpha$ is:
\begin{equation}
D_{\alpha}+h\sum_{j\in \mathcal{DS}_{\alpha}}C_{\alpha}^jW_jB_{\alpha}^j
\end{equation}
This is an $m_{\alpha}\times m_{\alpha}$ square matrix.
The extra-diagonal block term, in position ($\alpha,\beta$) is: 
\begin{equation}
h\sum_{j\in \mathcal{DS}_{\alpha}\cap\mathcal{DS}_{\beta}}C_{\alpha}^jW_jB_{\beta}^j
\end{equation}
and is a $m_{\alpha}\times m_{\beta}$ matrix. This matrix differs from 0 when interactions $\alpha$ and $\beta$ are coupled, ie have common DS. \\

Or, for the relation l of interaction $\alpha$, we get: 
\begin{equation}
D_{\alpha,l}+h\sum_{j\in \mathcal{DS}_{\alpha}}C_{\alpha,l}^jW_jB_{\alpha}^j
\end{equation}
for the diagonal, and 
\begin{equation}
h\sum_{j\in \mathcal{DS}_{\alpha}\cap\mathcal{DS}_{\beta}}C_{\alpha,l}^jW_jB_{\beta}^j
\end{equation}
for extra-diagonal terms. \\
$D_{\alpha,l}$, row number $l$ of $D_{\alpha}$, the same for $C_{\alpha,l}$


Finally, the linked-Interaction map provides, for each interaction (named ``current interaction''), the list of all the interactions (named ``linked interaction'') that have common dynamical system with the ``current interaction''.
\subsection{A simple example}

%We consider $n=5$ dynamical systems and $m=4$ interactions: 
%\begin{eqnarray*}
%I_{\mu}& \rightarrow& DS_1, DS_3, m_{\mu} = 3 \\
%I_{\theta}&\rightarrow& DS_3, DS_4, m_{\theta} = 1  \\
%I_{\gamma}&\rightarrow& DS_2,  m_{\gamma} = 1 \\
%I_{\zeta}&\rightarrow& DS_1, DS_5,  m_{\zeta} = 2 
%\end{eqnarray*}
%The linked-interaction map is :
%\begin{eqnarray*}
%I_{\mu} &\rightarrow& I_{\theta}, commonDS = DS_3 \\
%        &\rightarrow& I_{\zeta}, commonDS = DS_1 \\
%I_{\theta} &\rightarrow&I_{\mu}, commonDS = DS_3 \\
%I_{\zeta} &\rightarrow&I_{\mu}, commonDS = DS_1
%\end{eqnarray*}
%And:
%\begin{eqnarray*}
%M &=& 7, N = \displaystyle{\sum_{i=1}^{5} n_i} \\
%\mathcal{I}_1 &=& \{I_{\mu}, I_{\zeta} \}\\
%\mathcal{I}_2 &=& \{I_{\gamma}\} \\
%\mathcal{I}_3 &=& \{I_{\mu}, I_{\theta}\} \\
%\mathcal{I}_4 &=& \{I_{\theta} \} \\
%\mathcal{I}_4 &=& \{I_{\zeta}\}
%\end{eqnarray*}

We consider $n=3$ dynamical systems and $m=2$ interactions: 
\begin{eqnarray*}
I_{\mu}& \rightarrow& \mathcal{DS}_{\mu} = \{DS_1, DS_3\}, m_{\mu} = 3 \\
I_{\theta}&\rightarrow& \mathcal{DS}_{\theta} = \{DS_2, DS_3\}, m_{\theta} = 1  \\
\end{eqnarray*}
The linked-interaction map is :
\begin{eqnarray*}
I_{\mu} &\rightarrow& I_{\theta}, commonDS = DS_3 \\
I_{\theta} &\rightarrow&I_{\mu}, commonDS = DS_3 \\
\end{eqnarray*}
And:
\begin{eqnarray*}
M &=& 4, N = \displaystyle{\sum_{i=1}^{3} n_i} \\
\mathcal{I}_1 &=& \{I_{\mu} \}\\
\mathcal{I}_2 &=& \{I_{\theta}\} \\
\mathcal{I}_3 &=& \{I_{\mu}, I_{\theta}\} \\
\end{eqnarray*}

\begin{eqnarray}
y_1 = \left[\begin{array}{ccc} 
C_1^1 & C_1^3 \end{array}\right]
\left[\begin{array}{c}
x_1 \\
x_3 
\end{array}\right]
+ D_1\lambda_1 \\
y_2 = \left[\begin{array}{ccc} 
C_2^2 & C_2^3 \end{array}\right]
\left[\begin{array}{c}
x_2 \\
x_3 
\end{array}\right]
+ D_2\lambda_2 
\end{eqnarray}
%
\begin{eqnarray}
\left[\begin{array}{c}
R_1 \\
R_2 \\
R_3 \end{array}\right]=
\left[\begin{array}{c}
B_1^1\lambda_1  \\
B_2^2\lambda_2  \\
B_1^3\lambda_1 + B_2^3\lambda_2
\end{array}\right]
\end{eqnarray}
%
\begin{eqnarray}
M_{OSNSP} &=& \left[\begin{array}{cc} 
D_1+hC_1^1W_1B_1^1+hC_1^3W_3B_1^3 & hC_1^3W_3B_2^3 \\
hC_2^3W_3B_1^3 & D_2+hC_2^2W_2B_2^2+hC_2^3W_3B_2^3 
\end{array}\right]\left[\begin{array}{c} 
\lambda_1  \\
\lambda_2
\end{array}\right]_{k+1} 
\end{eqnarray}

\subsection{relative degree}
Let us consider the global vector 
\begin{eqnarray}
Y =\left[\begin{array}{c} 
y_1 \\
...  \\
y_M
\end{array}\right] = CX + D\Lambda
\end{eqnarray}
We denote by $r_j$ the relative degree of equation $j$, $j\in [1..M]$. 
We have:
\begin{eqnarray}
y_j = \displaystyle{\sum_{i=1}^n C_j^i x_i +D_{j,j}\lambda_j + \sum_{i\neq j, i=1}^m D_{j,i} \lambda_i } 
\end{eqnarray}
$D_{j,i}$ a scalar and $C_j^i$ a $1 \times n_i$ line-vector. \\
If $D_{jj} \neq 0$, then $r_j=0$. Else, we should consider the first derivative of $y_j$. \\
Before that, recall that: 
\begin{eqnarray}
R_i = \displaystyle{\sum_{k=1}^M B_k^i \lambda_j}
\end{eqnarray}
Through many of the $B_j^i$ are equal to zero, we keep them all in the following lines. \\
Then:

\begin{eqnarray}
\dot y_j &=& \displaystyle{\sum_{i=1}^n C_j^i (A_i x_i +  \sum_{k=1}^M B_k^i \lambda_k  ) + f(\lambda_k)_{k\neq j}} \\
&=& \displaystyle{\sum_{i=1}^n C_j^i (A_i x_i + B_j^i \lambda_j + \sum_{k=1,k\neq j}^M B_k^i \lambda_k  ) + \ldots}
\end{eqnarray}

So, if $\displaystyle{\sum_{i=1}^n C_j^i B_j^i} \neq 0$ (note that this corresponds to the product between line $j$ of $C$ and column $j$ of $B$) 
then $r_j=1$ else we consider the next derivative, and so on.  \\
In derivative $r$, the coefficient of $\lambda_j$ will be:
\begin{eqnarray}
coeff_j&=& \displaystyle{\sum_{i=1}^n C_j^i (A_i)^{r-1} B_j^i }
\end{eqnarray}
if $coeff_j\neq 0$ then $r_j = r$. 

%\subsection{Implementation}
%\begin{itemize}
%\item relative degree: function of $D,C,A$ off all interactions/relations, time invariant $\Rightarrow$ computed and saved in Topology.
%\item linkedInteractionMap $\Rightarrow$ computed and saved in Topology
%\item diagonal term: function of $D,C$ and $B$ of a specific interaction + function of $W$ of all DS concerned + time step. Time invariant.  \\
%$\Rightarrow$ compute and save in OSNSP, during initialize. 
%\item extra-diagonal terms: the same + depends of linkedInteractionMap $\Rightarrow$ compute and save in OSNSP, during initialize.  
%
% ============= LAGRANGIAN =====================
%
\section{LagrangianDS - Lagrangian Linear  Relations}
\subsection{General notations}
We consider $n$ dynamical systems, lagrangian and non linear, of the form: 
\begin{equation}
M_i(q_i) \ddot q_i + N_i(\dot q_i, q_i) = F_{Int,i}(\dot q_i , q_i , t)+F_{Ext,i}(t) + p_i
\end{equation}
Each system if of dimension $n_i$, and we denote $N = \displaystyle{\sum_{i=1}^{n} n_i}$. \\
An interaction, $I_{\alpha}$ is composed with a non smooth law, $nslaw_{\alpha}$ and a relation:
\begin{equation}
y_{\alpha} = H_{\alpha}Q_{\alpha} + b_{\alpha}
\end{equation}
The ``dimension'' of the interaction, ie the size of vector $y_{\alpha}$, is denoted $m_{\alpha}$ and we set: 
$$ M_y = \sum_{\alpha=1}^{m} m_{\alpha}$$
$m$ being the number of interactions in the Non Smooth Dynamical System.  \\
$Q_{\alpha}$ is a vector that represents the DS concerned by the interaction. Its dimension is noted $N_{\alpha}$, this for $n_{\alpha}$ systems in the interaction. \\
$H_{\alpha}$ is a $m_{\alpha} \times N_{\alpha}$ row-blocks matrix and $b_{\alpha}$ a $m_{\alpha}$ vector. \\
\begin{equation}
H_{\alpha}=\left[\begin{array}{ccc} 
H_{\alpha}^i & H_{\alpha}^j & ...\end{array}\right]
\end{equation}
with $i,j,...\in \mathcal{DS}_{\alpha}$ which is the set of DS belonging to interaction $\alpha$.\\
We also have the following relation: 
\begin{equation}
\left[\begin{array}{c} 
R_{\alpha}^i \\
R_{\alpha}^j \\
...  
\end{array}\right] = {}^tH_{\alpha}\lambda_{\alpha}
=\left[\begin{array}{c} 
{}^tH_{\alpha}^i \\
{}^tH_{\alpha}^j \\
...
\end{array}\right]\lambda_{\alpha}
\end{equation}
$R_{\alpha}^i$ represents the contribution of interaction $\alpha$ on the reaction of the dynamical system $i$, and ${}tH_{\alpha}^i$ is a $n_i \times m_{\alpha}$ block matrix. \\ 
And so: 
\begin{equation}
R_i = \sum_{\beta\in\mathcal{I}_i}R_{\beta}^i=\sum_{\beta\in\mathcal{I}_i}{}H^i_{\beta} \lambda_{\beta}
\end{equation}
with $\mathcal{I}_i$ the set of interactions in which dynamical system number $i$ is involved. \\
Introducing the time dicretisation, we get: 
\begin{eqnarray}
\dot q_i^{k+1} = \dot q_{free,i} + W_iR_i^{k+1}
\nonumber\\
\dot y_{\alpha}^{k+1} = H_{\alpha}\dot Q_{\alpha}^{k+1} \\
\nonumber\\
R_i^{k+1} = \sum_{\beta\in\mathcal{I}_i}H^i_{\beta} \lambda_{\beta}^{k+1}
\end{eqnarray}
ie, 
\begin{eqnarray}
  y_{\alpha}^{k+1} &=& H_{\alpha} Q_{\alpha}^{free} + H_{\alpha}W_{\alpha}{}^tH_{\alpha}\lambda_{\alpha}+\sum_{i\in \mathcal{DS}_{\alpha}}\sum_{\beta\in\mathcal{I}_i,\alpha\neq\beta}H_{\alpha}^iW_iH_{\beta}^j\lambda_{\beta}
\end{eqnarray}
with $W_{\alpha}$ given by \eqref{Walpha}. 

The global-assembled $Y$ vector, of dimension M, composed by $m$ $y_{\alpha}$ subvectors, is given by:
\begin{eqnarray}
Y_{k+1} = q_{OSNSP} + M_{OSNSP}\Lambda_{k+1}
\end{eqnarray}

with:
\begin{eqnarray}
q_{OSNSP}^{\alpha} = H_{\alpha} Q_{\alpha}^{free}
\end{eqnarray}
and for $M_{OSNSP}$, the block-diagonal term for block-row $\alpha$ is
\begin{equation}
\sum_{j\in \mathcal{DS}_{\alpha}}H_{\alpha}^jW_j{}^tH_{\alpha}^j
\end{equation}
an $m_{\alpha}\times m_{\alpha}$ square matrix.
The extra-diagonal block term, in position ($\alpha,\beta$) is: 
\begin{equation}
\sum_{j\in \mathcal{DS}_{\alpha}\cap\mathcal{DS}_{\beta}}H_{\alpha}^jW_j{}^tH_{\beta}^j
\end{equation}
and is a $m_{\alpha}\times m_{\beta}$ matrix. This matrix differs from 0 when interactions $\alpha$ and $\beta$ are coupled, ie have common DS. \\

Or, for the relation l of interaction $\alpha$, we get: 
\begin{equation}
\sum_{j\in \mathcal{DS}_{\alpha}}H_{\alpha,l}^jW_j{}^tH_{\alpha}^j
\end{equation}
for the diagonal, and 
\begin{equation}
\sum_{j\in \mathcal{DS}_{\alpha}\cap\mathcal{DS}_{\beta}}H_{\alpha,l}^jW_j{}^tH_{\beta}^j
\end{equation}
for extra-diagonal terms. \\
$H_{\alpha,l}$, row number $l$ of $H_{\alpha}$.


WARNING: depending on linear and non linear case for the DS, there should be a factor h ahead W. See Bouncing Ball template. 
\section{Block matrix approach}
The built of the OSNSProblem matrix could be computed using block
matrix structure. This section describe these matrices. It is not
implemented in Siconos.
Using previous notations, $n$ is the number of DS. $m$ the number of
interations.

\subsection{Block matrix of DS}
\[\boldsymbol{M}  \boldsymbol{\dot X}=\boldsymbol{A} \boldsymbol{X} + \boldsymbol{R}\]
where $\boldsymbol{M}=diag(M_1,...M_n)$ and
$\boldsymbol{A}=diag(A_1,..,A_n)$.
\[\boldsymbol{R}=\boldsymbol{B}\boldsymbol{\lambda}\]
\[\boldsymbol{B}=\left( \begin{array} {c} B^1_{1}...B^1_m\\.\\.\\
    B^n_1...B^n_m  \end{array}\right)\]
Some of $B^i_j$ doesn't exist.
\subsection{Block matrix of interaction}
\[ \boldsymbol{Y}= \boldsymbol{C}  \boldsymbol{X}+
\boldsymbol{D} \boldsymbol{\lambda}\]
with $ \boldsymbol{D}=diag(D_1..D_m)$ and 
\[ \boldsymbol{C}=\left( \begin{array} {c}
    C^1_{1}..C^n_1\\.\\.\\C^1_{m}...C^n_{m} \end{array}\right)\]
Some of $C^i_j$ doesn't exist.

\subsection{OSNSProblem using block matrices}
The Matrix of the OSNS Problem could be assembled using the following
block-product-matrices $\boldsymbol{C}\boldsymbol{W}\boldsymbol{B}$.
\chapter{Dynamical Systems formulations in Siconos.}

\begin{table}[!ht]
  \begin{tabular}{|l|l|}
    \hline
    author  & F. P\'erignon \\
    \hline
    date    & March 22, 2006 \\ 
    \hline
    version & Kernel 1.1.4 \\
    \hline
  \end{tabular}
\end{table}


\section{Class Diagram}
There are four possible formulation for dynamical systems in Siconos,
two for first order systems and two for second order Lagrangian systems. The main class is DynamicalSystem, all other derived from this one, as shown in the following diagram:
\begin{figure}[htbp]
  \centering
 \includegraphics[width=0.3\textwidth]{./DSClassDiagram.pdf}
  \label{DSDiagram}
\end{figure}
% DYNAMICAL SYSTEMS
\section{General non linear first order dynamical systems \\ $\rightarrow$ class \it{DynamicalSystem}}
This is the top class for dynamical systems. All other systems classes derived from this one. \\

A general dynamical systems is described by the following set of $n$ equations, completed with initial conditions:
\begin{eqnarray}
  \dot x &=& f(x,t) + T(x) u(x, \dot x, t) + r \\
  x(t_0)&=&x_0 
\end{eqnarray}

\begin{itemize}
\item $x$: state of the system - Vector of size $n$.
\item $f(x,t)$: vector field - Vector of size $n$.
\item $u(x, \dot x, t)$: control term - Vector of size $uSize$.
\item $T(x)$: $n\times uSize$ matrix, related to control term.
\item $r$: input due to non-smooth behavior - Vector of size $n$.
\end{itemize}

The Jacobian matrix, $\nabla_x f(x,t)$, of $f$ according to $x$, $n\times n$ square matrix, is also a member of the class. \\

Initial conditions are given by the member $x_0$, vector of size $n$. This corresponds to x value when
simulation is starting, ie after a call to strategy->initialize(). \\

There are plug-in functions in this class for $f$ (vectorField), $jacobianX$, $u$ and $T$. All
of them can handle a vector of user-defined parameters. 

% LINEAR DS
\section{First order linear dynamical systems $\rightarrow$ class \it{LinearDS}}

Derived from DynamicalSystem, described by the set of $n$ equations and initial conditions: 
\begin{eqnarray}
  \dot x &=& A(t)x(t)+Tu(t)+b(t)+r \\
  x(t_0)&=&x_0 
\end{eqnarray}
With:
\begin{itemize}
\item $A(t)$: $n\times n$ matrix, state independent but possibly time-dependent.
\item $b(t)$: Vector of size $n$, possibly time-dependent.
\end{itemize}
Other variables are those of DynamicalSystem class. \\
$A$ and $B$ have corresponding plug-in functions. \\

Warning: time dependence for $A$ and $b$ is not available at the time in the simulation part for this kind of dynamical systems. \\

Links with vectorField and its Jacobian are: 
\begin{eqnarray}
  f(x,t) &=& A(t)x(t)+b(t) \\
  jacobianX&=&\nabla_x f(x,t) = A(t) 
\end{eqnarray}

% LAGRANGIANDS
\section{Second order non linear Lagrangian dynamical systems \\  $\rightarrow$ class \it{LagrangianDS}}

Lagrangian second order non linear systems are described by the following set of$nDof$ equations + initial conditions:
\begin{eqnarray}
 M(q) \ddot q + NNL(\dot q, q) + F_{Int}(\dot q , q , t) &=& F_{Ext}(t) + p \\
 q(t_0) &=& q0 \\
 \dot q(t_0) &=& velocity0 
\end{eqnarray}
With:
\begin{itemize}
\item $M(q)$: $nDof\times nDof$ matrix of inertia.
\item $q$: state of the system - Vector of size $nDof$.
\item $\dot q$ or $velocity$: derivative of the state according to time - Vector of size $nDof$.
\item $NNL(\dot q, q)$:  non linear terms, time-independent - Vector of size $nDof$.
\item $F_{Int}(\dot q , q , t)$: time-dependent linear terms - Vector of size $nDof$.
\item $F_{Ext}(t)$: external forces, time-dependent BUT do not depend on state - Vector of size $nDof$.
\item $p$: input due to non-smooth behavior - Vector of size $nDof$.
\end{itemize}

The following Jacobian are also member of this class:
\begin{itemize}
\item jacobianQFInt = $\nabla_q F_{Int}(t,q,\dot q)$ - $nDof\times nDof$ matrix.
\item jacobianVelocityFInt = $\nabla_{\dot q} F_{Int}(t,q,\dot q)$ - $nDof\times nDof$ matrix.
\item jacobianQNNL = $\nabla_q NNL(q,\dot q)$ - $nDof\times nDof$ matrix.
\item jacobianVelocityNNL = $\nabla_{\dot q}NNL(q,\dot q)$ - $nDof\times nDof$ matrix.
\end{itemize}


There are plug-in functions in this class for $F_{int}$, $F_{Ext}$, $M$, $NNL$ and the four Jacobian matrices. All
of them can handle a vector of user-defined parameters. \\

Links with first order dynamical system are: 
\begin{eqnarray}
  n &= &2nDof \\
  x &=&\left[\begin{array}{c}q \\ \dot q \end{array}\right] \\
  f(x,t) &=&  \left[\begin{array}{c} \dot q \\ M^{-1}(F_{Ext}-F_{Int}-NNL) \end{array}\right] \\
  \\
  \nabla_x f(x,t) &=& 
  \left[\begin{array}{cc} 
      0_{nDof\times nDof} & I_{nDof\times nDof} \\
      \nabla_q(M^{-1})(F_{Ext}-F_{Int}-NNL) -M^{-1}\nabla_q(F_{Int}+NNL) &  -M^{-1}\nabla_{\dot q}(F_{Int}+NNL) 
    \end{array}\right] \\
  r &=& \left[\begin{array}{c} 0_{nDof} \\ p \end{array}\right] \\
  u(x,\dot x,t) &=& u_L(\dot q, q, t) \text{  (not yet implemented)} \\
  T(x) &=& \left[\begin{array}{c} 0_{nDof} \\ T_L(q) \end{array}\right] \text{  (not yet implemented)} \\
\end{eqnarray}
With $0_{n}$ a vector of zero of size $n$, $0_{n\times m}$ a $n\times m$ zero matrix and
$I_{n\times n}$, identity $n\times n$ matrix. \\

Warning: control terms ($Tu$) are not fully implemented in Lagrangian systems. This will be part of future version.

% LAGRANGIAN LINEAR TIME INVARIANT DS
\section{Second order linear and time-invariant Lagrangian dynamical systems $\rightarrow$ class \it{LagrangianLinearTIDS}}
\label{Sec:LagrangianLineatTIDS}
\begin{eqnarray}
M \ddot q + C \dot q + K q =  F_{Ext}(t) + p
\end{eqnarray}

With:
\begin{itemize}
\item $C$: constant viscosity $nDof\times nDof$ matrix 
\item $K$: constant rigidity $nDof\times nDof$ matrix 
\end{itemize}

And: 
\begin{eqnarray}
F_{Int} &=& C \dot q + K q \\
NNL &=& 0_{nDof} 
\end{eqnarray}



\chapter{Dynamical Systems implementation in Siconos.}

\begin{table}[!ht]
  \begin{tabular}{|l|l|}
    \hline
    author  & F.  P\'erignon \\
    \hline
    date    & November 7, 2006 \\ 
    \hline
    version & Kernel 1.3.0 \\
    \hline
  \end{tabular}
\end{table}




\section{Introduction}
This document is only a sequel of notes and remarks on the way Dynamical Systems are implemented in Siconos.\\
It has to be completed, reviewed, reorganized etc etc for a future Developpers'Guide. \\
See also documentation in Doc/User/DynamicalSystemsInSiconos for a description of various dynamical systems types.

\section{Class Diagram}
There are four possible formulation for dynamical systems in Siconos,
two for first order systems and two for second order Lagrangian systems. The main class is DynamicalSystem, all other derived from this one, as shown in the following diagram:
\begin{figure}[htbp]
  \centering
 \includegraphics[width=0.3\textwidth]{./DSClassDiagram.pdf}
  \label{DSDiagram}
\end{figure}
% DYNAMICAL SYSTEMS

\section{Construction}

Each constructor must:
\begin{itemize}
\item initialize all the members of the class and of the top-class if it exists
\item allocate memory and set value for all required inputs
\item allocate memory and set value for optional input if they are given as argument (in xml for example)
\item check that given data are coherent and that the system is complete (for example, in the LagrangianDS
if the internal forces are given as a plug-in, their Jacobian are also required. If they are not given, this leads to an exception).
\end{itemize}

No memory allocation is made for unused members $\Rightarrow$ requires if statements in simulation.  (if!=NULL ...).\\

\subsection{DynamicalSystem}

{\bf Required data:}\\
n, x0, f, jacobianXF \\
{\bf Optional:}\\
T,u \\

\textbf{Always allocated in constructor:} \\
x, x0, xFree, r, rhs, jacobianXF

Warning: default constructor is always private or protected and apart from the others and previous rules or remarks do not always apply to it. 
This for DS class and any of the derived ones. 

\subsection{LagrangianDS}

\textbf{Required data:}\\
ndof, q0, velocity0, mass \\
\textbf{Optional:}\\
fInt and its Jacobian, fExt, NNL and its Jacobian. \\

\textbf{Always allocated in constructor:} \\
mass, q, q0, qFree, velocity, velocity0, velocityFree, p. \\
All other pointers to vectors/matrices are set to NULL by default. \\
Memory vectors are required but allocated during call to initMemory function. 

Various rules:
\begin{itemize}
\item fInt (NNL) given as a plug-in $\Rightarrow$ check that JacobianQ/Velocity are present (matrices or plug-in)
\item any of the four Jacobian present $\Rightarrow$ allocate memory for block-matrix jacobianX  (connectToDS function)
\item 
\end{itemize}

check: end of constructor or in initialize? \\
computeF and JacobianF + corresponding set functions: virtual or not? \\


\section{Specific flags or members}

\begin{itemize}
\item isAllocatedIn: to check inside-class memory allocation
\item isPlugin: to check if operators are computed with plug-in or just directly set as a matrix or vector
\item workMatrix: used to save some specific matrices in order to avoid recomputation if possible (inverse of mass ...)
\end{itemize}

\section{plug-in management}
DynamicalSystem class has a member named parameterList which is a $map<string, SimpleVector*>$, ie a list of
pointers to SimpleVector*, with a string as a key to identified them. 
For example, $parametersList["mass"]$ is a SimpleVector*, which corresponds to the last argument given in 
mass plug-in function. \\
By default, each parameters vectors must be initialized with a SimpleVector of size 1, as soon as the plug-in is
declared. Moreover, to each vector corresponds a flag in isAllocatedIn map, to check if the corresponding vector has been 
allocated inside the class or not. \\ 
For example, in DynamicalSystem, if $isPlugin["vectorField"]==true$, then, during call to constructor or set function,
it is necessary to defined the corresponding parameter: \\
$parametersList["vectorField"] = new SimpleVector(1)$ \\
and to complete the $isAllocatedIn$ flag: \\
$isAllocatedIn["parameter_for_vectorField"] = true$. \\

\chapter{Interactions}
\begin{table}[!ht]
  \begin{tabular}{|l|l|}
    \hline
    author  & F.  P\'erignon \\
    \hline
    date    & November 7, 2006 \\ 
    \hline
    version & Kernel 1.3.0 \\
    \hline
  \end{tabular}
\end{table}

\section{Introduction}
This document is only a sequel of notes and remarks on the way Interactions are implemented in Siconos.\\
It has to be completed, reviewed, reorganized etc etc for a future Developpers'Guide. \\
See also documentation in Doc/User/Interaction.

\section{Class Diagram}

\section{Description}

\begin{ndrfp} 
review of interactions (for EventDriven implementation) 17th May 2006.
\end{ndrfp}

\bei
\item variable \varcpp{nInter} renamed in \varcpp{interactionSize}: represents the size of \varcpp{y} and \varcpp{$\lambda$}. NOT the number of relations !! \\
\item add a variable \varcpp{nsLawSize} that depends on the non-smooth law type.\\
Examples:
\bei
\item NewtonImpact -> \varcpp{nsLawSize} = 1
\item Friction 2D  -> \varcpp{nsLawSize} = 2
\item Friction 3D  -> \varcpp{nsLawSize} = 3
\item ... 
\item \varcpp{nsLawSize} = n with n dim of matrix D in :
$y=Cx+D\lambda$, D supposed to be a full-ranked matrix. \\
Warning: this case is represented by only one relation of size n. 
\ei
\item \varcpp{numberOfRelations}: number of relations in the interaction, \varcpp{numberOfRelations} = $\Frac{\varcpp{interactionSize}}{\varcpp{nsLawSize}}$.
\ei


\chapter{Notes on the Non Smooth Dynamical System construction}
\begin{table}[!ht]
  \begin{tabular}{|l|l|}
    \hline
    author  & F.  P\'erignon \\
    \hline
    date    & November 7, 2006 \\ 
    \hline
    version & Kernel 1.3.0 \\
    \hline
  \end{tabular}
\end{table}

\section{Introduction}

\section{Class Diagram}

\section{Description}

Objects must be constructed in the following order: 
\bei
\item DynamicalSystems
\item NonSmoothLaw: depends on nothing
\item Relation: no link with an interaction during construction, this will be done during initialization. 
\item Interaction: default constructor is private and copy is forbidden. Two constructors: xml and from data. Required data are a DSSet, a NonSmoothLaw and
a Relation (+ dim of the Interaction and a number). \\
Interaction has an initialize function which allocates memory for y and lambda, links correctly the relation and initializes it .... This function is called at the 
end of the constructor. That may be better to call it in simulation->initialize? Pb: xml constructor needs memory allocation for y and lambda if they are
provided in the input xml file. 
\item NonSmoothDynamicalSystem: default is private, copy fobidden. Two constructors: xml and from data. Required data are the DSSet and the InteractionsSet.
The topology is declared and constructed (but empty) during constructor call of the nsds, but initialize in the Simulation, this because it can not be initialize until the nsds has been fully described (ie this to allow user to add DS, Inter ...) at any time in the model, but before simulation initialization). 

\ei

\section{misc}

\bei 
\item no need to keep a number for Interactions? Only used in xml for OSI, to know which Interactions it holds.
\item pb: the number of saved derivatives for y and lambda in Interactions is set to 2. This must depends on the relative degree which is computes during
Simulation initialize and thus too late. It is so not available when memory is allocated (Interaction construction). Problem-> to be reviewed.
\ei 


\chapter{OneStepIntegrator and derived classes.}
\begin{table}[!ht]
  \begin{tabular}{|l|l|}
    \hline
    author  & F.  P\'erignon \\
    \hline
    date    & November 7, 2006 \\ 
    \hline
    version & Kernel 1.3.0 \\
    \hline
  \end{tabular}
\end{table}

\section{Introduction}
This document is only a sequel of notes and remarks on the way OneStepIntegrators are implemented in Siconos.\\
It has to be completed, reviewed, reorganized etc etc for a future Developpers'Guide. \\
See also documentation in Doc/User/OneStepIntegrator for a description of various OSI.

\section{Class Diagram}

\section{Misc}

OSI review for consistency between Lsodar and Moreau:
\begin{itemize}
\item add set of DynamicalSystem*
\item add set of Interaction* 
\item add link to strategy that owns the OSI
\item remove td object in OSI -> future: replace it by a set of td (one per ds)
\item add strat in constructors arg list
\end{itemize}



osi -> strat -> Model -> nsds -> topology \\
osi -> strat -> timeDiscretisation \\

let a timeDiscretisation object in the OSI? set of td (one per ds)? \\
create a class of object that corresponds to DS on the simulation side ? \\
will contain the DS, its discretization, theta for Moreau ... ? \\ 
Allow setStrategyPtr operation? Warning: need reinitialisation. \\


Required input by user: \\
\begin{itemize}
\item list of DS or list of Interactions ? 
\item pointer to strategy
\item ...
\end{itemize}

\section{Construction}

Each constructor must:

\begin{itemize}
\item
\end{itemize}

\subsection{Moreau}

Two maps: one for W, and one for theta. To each DS corresponds a theta and a W. \\
Strategy arg in each constructor.

\textbf{Required data:}\\

\textbf{Optional:}\\

\textbf{Always allocated in constructor:} \\

Warning: default constructor is always private or protected and apart from the others and previous rules or remarks do not always apply to it. 

\subsection{Lsodar}

\textbf{Required data:}\\

\textbf{Optional:}\\

\textbf{Always allocated in constructor:} \\

\chapter{First Order Nonlinear Relation }

\begin{table}[!ht]
  \begin{tabular}{|l|l|}
    \hline
    author  & 0. Bonnefon \\
    \hline
    date    & July, 1 2009 \\ 
    \hline
    version & Kernel 3.0.0 \\
    \hline
  \end{tabular}
\end{table}

\chapter{Computation of the number of Index Set and various levels}
\begin{table}[!ht]
  \begin{tabular}{|l|l|}
    \hline
    author  & V. Acary \\
    \hline
    date    & Septembre 16, 2011 \\ 
    \hline
    version & Kernel 3.3.0 \\
    \hline
  \end{tabular}
\end{table}

\input{ChapterLevelsAndIndices.tex}


\chapter{Newton's linearization for First Order Systems}
\input{chapterNewton.tex}
\chapter{Newton's linearization for Lagrangian systems}
\input{chapterNewtonLagrangian.tex}
\chapter{NewtonEuler Dynamical Systems}
\input{NewtonEuler.tex}
\chapter{NewtonEulerR: computation of $\nabla _qH$}
\input{NewtonImpactJacobian.tex}
\chapter{Projection On constraints}
\input{ProjOnConstraints.tex}

%\begin{figure}[htbp]
%  \input{./Figures/Diagram.pstex_t}
%\end{figure}


\chapter{Simulation of a Cam Follower System}
{\textbf Main Contributors:} {\textit{Mario di Bernardo, Gustavo Osorio, Stefania Santini}}\\
\textit{University of Naples Federico II, Italy.}\\

\input{./CamFollower/Example_Manual_Cam_Follower}
\chapter{Quartic Formulation}
\input{QuarticFormulation.tex}


\chapter{Alart--Curnier Formulation}
\input{AlartCurnier.tex}

\bibliographystyle{plain}
\bibliography{bibli}
\end{document}


%%% Local Variables:
%%% mode: latex
%%% TeX-master: t
%%% End:
